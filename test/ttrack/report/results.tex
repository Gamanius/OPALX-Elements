\section{Simulations for SLS}
Using a linear lattice for the SLS, we find for different momenta acceptance the following lifetimes with $6 \cdot 10^9$ particles in the bunch, an RF Voltage of 2.1 kV and an Energy of 2.4 GeV.
\begin{figure}[here]
\centering
\begin{tabular}{l|l}
Momenta Acceptance & Touschek Lifetime \\ \hline
0.5 \% & 0.12 h \\
1.0 \% & 1.01 h \\
1.5 \% & 3.70 h \\
2.0 \% & 8.71 h 
\end{tabular}
\caption{Touschek Lifetime as a function of Momenta Acceptance}
\end{figure}
These computations were done using a matched beam comparable to the current one in the SLS.
\begin{figure}[here]
 \centering
   \includegraphics[width=0.80\textwidth]{sls-acceptance.pdf}
 \caption{SLS Acceptance vs Lifetime}
\end{figure}
\subsection{Simulation vs. Experiment}
As you can see on the figure, the simulated Touschek Lifetimes are longer then the ones in the experiment. This can be due to several effects, as changes in the bunch shape due to non-linear effects, like the quadruples, or shortening from the lifetime caused by scattering at the residual gas.\\
Another reason is that cutting into the momentum acceptance using a scrapper only reduces the momentum acceptance at one place. For the momentum acceptance computed using our program we assume, that it is the same everywhere.
\section{Low Emittance studies}
On the ``Workshop on small emittance lattices'' in March 2004,  Mikael Eriksson speculated that the Touschek Lifetime goes up as the emittance goes down. This is what we are studying here.For the emittance $\epsilon$ of the beam, we have:
\begin{equation} \epsilon \rightarrow \frac \epsilon F \end{equation}
Where F is factor greater then one. We are scaling the bunch geometry, as follows:
\begin{eqnarray}
\sigma_x &\rightarrow& \frac{\sigma_x}{\sqrt F} \\
\sigma_{x'} &\rightarrow& \frac{\sigma_{x'}}{\sqrt F} .
\end{eqnarray}
The justification for this, is that $\sigma_x = \sqrt{\epsilon_x \cdot \beta_x}$, and our $\beta$-function stays constant as we change the emittance.
\begin{figure}[here]
 \centering
   \includegraphics[width=0.80\textwidth]{emi.pdf}
 \caption{Low Emittance Studies for SLS} 
\end{figure}
This figure confirms what Michael Eriksson assumed, the Touschek Lifetime goes up, with ultralowemittance.\\
Values for ultralowemittance show bad convergence behavior, because the event number is low. However, one event counts a lot, because the densities are very high.
\subsection{Comparison with the classical formula}
For the Aurora lattice with its high symmetry, we can run both the classical formula. We have done this, as you can see on the next figure.
\begin{figure}[here]
 \centering
   \includegraphics[width=0.80\textwidth]{emiaur.pdf}
 \caption{Low Emittance Studies for Aurora}
\end{figure}
You can see that, till a factor $F = 10^4$ the classical formula and our simulation predict the same behavior, but then the classical formula goes up much more quickly.
