\documentclass[11pt,a4paper]{article}
\usepackage{amsmath}
\usepackage{amsfonts}
\usepackage{graphicx}
\usepackage{url}
\usepackage{cite}
\usepackage{algorithm,algorithmic}

\textheight 9in \textwidth 6.5in
\topmargin -.2in \oddsidemargin 0cm

\renewcommand {\Re}{{\mathbb{R}}}
\newcommand {\Co}{{\mathbb{C}}}
%%
\renewcommand {\div}{{\mbox{div}\:}}
\newcommand {\abs}[1]{\lvert#1\rvert}
\newcommand {\norm}[1]{\lVert#1\rVert}
%%
\newcommand {\Hcurl}{{H(\mathbf{curl};\Omega)}}
\newcommand {\Hocurl}{{H_0(\mathbf{curl};\Omega)}}
\newcommand {\Hdiv}{{H(\mathbf{div};\Omega)}}
\newcommand {\Hodiv}{{H_0(\mathbf{div};\Omega)}}
%%
\newcommand {\grad}{{\mbox{\textbf{grad}}\:}}
\newcommand {\curl}{{\mbox{\textbf{curl}}\:}}
\renewcommand {\div}{{\mbox{div}\:}}
%%

\begin{document}

\setcounter{section}{1}
\section{Scientific Information on Project \\
  ``Adaptive Mesh Refinement''%
}

\begin{center}
  Peter Arbenz, Chair of Computational Science, ETH, Zurich \\
  Andreas Adelmann, Large Research Facilities Division (GFA), Paul Scherrer
  Institut, Villigen
\end{center}

{\small
  \subsection{Summary}
  
  Resonant electromagnetic cavity structures are used in virtually all
  types of particle accelerators.  The X-ray free electron laser
  currently under study at the Paul Scherrer Institut, is no exception
  and will consist of a large variety of radio frequency (RF) structures
  for guiding and accelerating electrons from the photo-cathode through
  the linear accelerator section.
  %%
  The numerical computation of eigenfrequencies and corresponding
  eigenmodal fields of large accelerator cavities, based on full-wave,
  three-dimensional models, has therefore attracted considerable
  interest in the recent past.
  %%
  Given the fact that the manufacturing of a cavity is expensive and
  time-consuming, a device that does not comply with its specifications
  cannot be tolerated.
  %%
  A robust and reliable pre-fabrication computer simulation is therefore
  indispensable.
  %%
  This problem has been addressed extensively in the past.
  %%
  Much work has been invested to compute electromagnetic eigenmodes.
  %%
  %%
  In many cases the eigenproblem has been modeled without electromagnetic
  loss mechanisms.
  %%
  This is certainly a justifiable approximation, especially when saving
  computational expense is an issue or when the cavity becomes
  large in terms of the dominant wavelength.
  %%
  %%
  However, the eigenmodal solution is affected considerably
  by loss. Therefore it must be considered for designing novel
  types of cavities.
  %%
  Traditionally, loss has been integrated into the model by computing
  the quality factor $Q$.
  %%
  The electromagnetic power dissipated in the cavity boundary is
  calculated with a perturbation approach using the magnetic field
  $\mathbf{B}$ \emph{after} the eigenmodes have been computed.
  %%
  While this allows for estimating that important cavity parameter
  it does not model the effect of loss onto the resonance frequency
  which is one of the most important cavity parameters.
  %%
  Only if losses are integrated into the eigenproblem \emph{a
    priori} can we extract thereby affected resonance frequencies from
  the eigensolution.
  %%
  There is some work into this direction.  We note that often the
  approaches are  relatively small with only a
  few thousand unknowns, or, losses, introduced either by holes in the
  aperture or cavity walls with a 
  finite conductivity, are not addressed.
  %%
  Today's accelerator cavities require a computational
  mesh with tens of millions of unknowns and, ultimately, in the region
  of $10^{8}$ unknowns.
  %%
  From the related field of microwave antenna design based on the
  microstrip principle and dielectric resonator technology it is known
  that, when modeling dielectric and ohmic losses, the antenna's
  resonance frequency is displaced significantly, i.e., by a few
  percent, when compared to a model without losses which is a serious
  modeling deficiency.
  %%
  In the field of accelerator cavity designs accuracy is quintessential,
  both for the absolute position of the resonance frequency and its relative
  displacement due to tuning mechanisms.
  %%
  Today's hardware offers significantly more
  memory and computing power.
  %%
  Therefore, the analysis of loss-plagued electromagnetic eigenvalue
  problems becomes feasible.
  %%
  This research proposal addresses the solution of very large
  electromagnetic eigenmodal problems as they are \emph{currently}
  encountered in accelerator design and as they will be in the
  foreseeable future.
  %%
  We intend to used the finite element (FE) method because
  modern cavity designs typically exhibit delicate and
  detailed geometrical features that must be considered for obtaining
  accurate results.
  %%
  We furthermore intend to model different loss mechanisms:
  %%
  these include
  %%
  (1) lossy dielectric and/or magnetic materials;
  %%
  (2) larger or smaller apertures in the boundary of the resonating
  structure which significantly change the character of the eigenvalue
  problem by introducing loss caused by electromagnetic fields radiating
  away part of the energy stored in the field;
  %%
  (3) finite conductivity cavity wall by the inclusion of a surface
  impedance boundary condition.
  
  We plan to extend an available parallel solver for large-scale real
  symmetric eigenvalue problems into a solver of complex symmetric
  eigenvalue problems.
  Jacobi--Davidson-type solver are the least memory consuming algorithms
  and therefore the ones best suited for tackling problems with tens or
  even hundreds of millions of degrees of freedom.  A crucial issue for
  parallel efficiency will be to find a reasonably scalable
  preconditioner for the correction equation of the Jacobi--Davidson
  algorithm.  We will base our implementation on the
  Trilinos framework that provides an
  infrastructure that is well suited for storing parallel objects like
  (multi-)vectors and sparse matrices.  Trilinos also provides solver
  suites and numerous preconditioners.
}

%%\newpage

%%%%%%%%%%%%%%%%%%%%%%%%%%%%%%%%%%%%%%%%%%%%%%%%%%%%%%%%%%%%%%%%%%%%%%%%

\subsection{Research plan}
\label{sec:research_plan}

\subsubsection{Account of the state of research in the field}
\subsubsection*{Institute of Computational Science}
\subsubsection*{Large Research Facilities Division (GFA), Paul Scherrer
  Institut}

%%%%%%%%%%%%%%%%%%%%%%%%%%%%%%%%%%%%%%%%%%%%%%%%%%%%%%%%%%%%%%%%%%%%%%%

\subsection{Detailed research plan}

\subsubsection*{Scientific and technical description}


\subsection{Timetable, milestones and deliverables}

The project is planned to last 3 years.  The various phases are
summarized in the following table.
%%
\begin{center}
  \begin{tabular}{|c|c|l|}
    \hline
    Phase & Duration & Work \\  \hline
    1 & 6 months &  Initial skill adaptation training \\
    2 & 12 months & Make \textsf{femaxx} a complex symmetric solver \\
    3 & 3 months & Extend for quadratic eigenvalue problem \\
    4 & 9 months & Extend for nonlinear eigenvalue problem \\
    5 & 6 months & Write thesis and wrap up \\  \hline
  \end{tabular}
\end{center}

%%%%%%%%%%%%%%%%%%%%%%%%%%%%%%%%%%%%%%%%%%%%%%%%%%%%%%%%%%%%%%%%%%%%%%%

\subsection{Significance of the planned research to the scientific
  community and to eventual potential users}
\label{sec:sign}

%%%%%%%%%%%%%%%%%%%%%%%%%%%%%%%%%%%%%%%%%%%%%%%%%%%%%%%%%%%%%%%%%%%%%%%

\subsection{Significance with respect to scientific, commercial, and
  other application areas}
\label{sec:sign2}

\nocite{hens:05,cale:05,gbgk:05,vslk:09}

\bibliography{amr,bib}
\bibliographystyle{plain}

\end{document}
