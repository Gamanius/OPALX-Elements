\documentclass[10pt,a4paper]{scrartcl}

\usepackage{iff}
\usepackage{array}
\usepackage{amsmath}
\usepackage{url}  
\usepackage{tikz}
\usepackage{pgflibraryshapes}  
\usetikzlibrary{arrows}
\usetikzlibrary{mindmap,trees}
\usepackage[overlay,absolute]{textpos}

\usepackage[latin1]{inputenc}
\usepackage[english]{babel}
\usepackage{pgfpages}

\newcommand{\ippl}{\textsc{IP$^2$L}}

\begin{document}

\title{\ippl~ Improvements: \\ \texttt{Update}, \texttt{CachedParticleLayout}
and P$^3$M}
\author{A.~Adelmann, Y.~Ineichen\\ 
        \url{{andreas.adelmann, yves.ineichen}@psi.ch} \\
        \small{Paul Scherrer Institut, Villigen, Switzerland}
}
\date{\today}
\maketitle


\section{Introduction}

\ippl~ (Independent Parallel Particle Layer) is an object-oriented framework for
particle based applications in computational science requiring high-performance
parallel computers. One of \ippl's most attractive features is its high
performance on both single-processor and distributed-memory multicomputers
machines. As future releases of the library will also support shared-memory
multicomputers, \ippl's authors have had to think very carefully about how to
obtain the best possible performance across a wide range of applications on
different architectures.

As with all big software frameworks maintainability is a difficult task. In the
last couple of years some deeply anchored issues resurfaced, e.g. the particle
update mechanism seems to have problems dealing with particles located directly
on a processor boundary. Unfortunately barely no man power could be spared to
properly address these issues.

This project aims to get rid of remaining issues and include new feature arising
in state-of-the-art problems, maintaining \ippl's usefulness in current and future
applications.

\section{Goals}

This project consists of tree goals, one fix (particle update) and two new
features (particle cached layout and P$^3$M (Particle-Particle-Particle Mesh)).
At least the first two tasks should be achievable in the available 3 months.

\subsection{\texttt{Update}}

\ippl's update function is responsible for handling all the administrative stuff
introduced by the parallel nature of the code. It ensures that all particles are
on the rightful processor (after having moved around) and updates (if necessary)
topology information.

As mentioned there are a couple of issues with the current version of the
\texttt{Update} method:
%
\begin{itemize}
    \item particles on a processor boundary have to be ``nudged'' into one of
    the two processors domains
    \item the communication scheme is not always optimal (MPI Buffer problems in
    some cases)
    \item proper handling of processors that do not possess any particle at all
\end{itemize}
%
To remedy this situation we propose the following:
%
\begin{itemize}
    \item review methods: update, binary repartition and boundp
    \item draw flow diagrams explaining the application flow of these methods to
    establish understanding
    \item investigate problems causing cores with 0 particles to crash
    \item rewrite update: fixing all remaining issues, plan how to
    handle different communication scheme
    \item update \ippl~ manual
\end{itemize}


\subsection{\texttt{CachedParticleLayout}}

In some application it is useful to have knowledge of particles contained on
neighboring processors. There is already a \texttt{ParticleInteractLayout},
containing all particles within a sphere of radius $r$, available. The
\texttt{CachedParticleLayout} is just another interpretation of the
\texttt{ParticleInteractLayout} with a different geometrical shape (box instead
of sphere).

There already exists a very crude adaption of the interact layout.
Unfortunately, this has never been tested. The tasks contain of the following
item:
%
\begin{itemize}
    \item code review of cached particle layout
    \item test and fix or rewrite cached particle layout
    \item in case of rewrite: class abstraction where any geometrical ``caching
    shape'' can be specified..
    \item update \ippl~ manual
\end{itemize}


\subsection{P$^3$M}

If there is still time we would like to add a P$^3$M code. This will improve
situation where we have to take into account multiple resolutions to study
effects on different scales, e.g., study micro-bunch instabilities.

\TODO[add]{LITERATURE}
\begin{itemize}
    \item \url{http://en.wikipedia.org/wiki/P3M}
    \item \url{http://arxiv.org/abs/astro-ph/9805096}
    \item \url{http://arxiv.org/abs/astro-ph/0512030}
\end{itemize}




\section{Timetable}

In Table \ref{tbl:time} we propose a timetable.

\TODO[FINISH]{Timetable}
\begin{table}[ht]\footnotesize
\begin{center}  
\begin{tabular}{lclllll} 
\hline 
\bf & \bf Week  & \bf Activity & \bf Milestone  \\
\hline \\
%\hline
1 & 1-2 & review update/repart/bound  & gain understanding of underlaying \\
  &     & and draw flow diagrams      & mechanisms of these important IPPL methods. \\
\\
\hline 
\end{tabular} 
\caption{Time schedule of the proposed project}
\label{tbl:time}
\end{center}
\end{table}

\end{document}
