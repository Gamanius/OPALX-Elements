\documentclass[10pt,a4paper]{scrartcl}

\usepackage{iff}
\usepackage{array}
\usepackage{amsmath}
\usepackage{url}  
\usepackage{tikz}
\usepackage{pgflibraryshapes}  
\usetikzlibrary{arrows}
\usetikzlibrary{mindmap,trees}
\usepackage[overlay,absolute]{textpos}

\usepackage[latin1]{inputenc}
\usepackage[english]{babel}
\usepackage{pgfpages}

\newcommand{\ippl}{\textsc{IP$^2$L}}

\begin{document}

\title{\ippl~ Improvements: \\ \texttt{Update}, \texttt{CachedParticleLayout}
and P$^3$M}
\author{A.~Adelmann, Y.~Ineichen\\ 
        \url{{andreas.adelmann, yves.ineichen}@psi.ch} \\
        \small{Paul Scherrer Institut, Villigen, Switzerland}
}
\date{\today}
\maketitle


\section{Introduction}

\ippl~ (Independent Parallel Particle Layer) is an object-oriented framework for
particle based applications in computational science requiring high-performance
parallel computers. One of \ippl's most attractive features is its high
performance on both single-processor and distributed-memory multicomputers
machines. As future releases of the library will also support shared-memory
multicomputers. The \ippl's authors have had to think very carefully about how to
obtain the best possible performance across a wide range of applications on
different architectures.

As with all big software frameworks maintainability is a difficult task. In the
last couple of years some deeply anchored issues resurfaced, e.g. the particle
update mechanism seems to have problems dealing with particles located directly
on a processor boundary and the framework dislikes a zero count of particles on a computational node.

% Unfortunately barely no man power could be spared to properly address these issues.

This project aims to get rid of these issues and include new feature arising
in state-of-the-art problems, maintaining \ippl's usefulness in current and future
applications.

\section{Goals}

This project consists of tree parts:
\begin{enumerate}
\item fix CIC interpolation and particle update 
\item commission the particle cached layout
\item implement, validate and benchmark the P$^3$M (Particle-Particle-Particle Mesh) method.
\end{enumerate}
%At least the first two tasks should be achievable in the available 3 months.

\subsection{CIC \& \texttt{Update}}

\subsubsection{\texttt{CIC}}
We just recently found out that the CIC interpolation has maybe a problem. See tests done by Yves.
\subsubsection{\texttt{Update}}

\ippl's \texttt{update} function is responsible for handling the book keeping w.r.t. the parallel nature of the framework.  It mainly ensures that all particles are
on the right core (after having moved) and updates (if necessary)
topology information.

As mentioned there are a couple of issues with the current version of the
\texttt{Update} method:
%
\begin{itemize}
    \item particles on a processor boundary have sometimes to be ``nudged'' into the proper domain
    \item the communication scheme is not always optimal (MPI Buffer problems)
    \item proper handling of cores that do not possess any particle at all
\end{itemize}
%
To remedy this situation we propose the following:
%
\begin{itemize}
    \item review the following methods: \texttt{update}, \texttt{binary repartition} and \texttt{boundp}
    \item draw flow diagrams explaining these methods and hence
    establish precise understanding
    \item investigate problems causing cores with 0 particles to crash
    \item rewrite update: fixing all remaining issues, plan how to
    handle different communication scheme
    \item update \ippl~manual
\end{itemize}


\subsection{\texttt{CachedParticleLayout}}

In some application it is useful to have knowledge of particles contained on
neighboring processors. There is already a \texttt{ParticleInteractLayout},
containing all particles within a sphere of radius $r$, available. The
\texttt{CachedParticleLayout} is just another interpretation of the
\texttt{ParticleInteractLayout} with a different geometrical shape (box instead
of sphere).

There already exists a preliminary implementation of the \texttt{CachedParticleLayout}.
A comprehensive test has unfortunately never been performed. This tasks contain of the following
item:
%
\begin{itemize}
    \item code review of \texttt{CachedParticleLayout}
    \item do a comprehensive test and fix upcoming issues
    \item in case of a rewrite: study a class abstraction where any geometrical ``caching 
    shape'' can be specified. 
    \item update \ippl~manual
\end{itemize}


\subsection{P$^3$M}

We would like to add a P$^3$M code. This will increase the accuracy of the field calculation by
removing the resolution limit given by the mesh size. In the SwissFEL project, this will allow us
to study micro-bunch instabilities on scales not access-able before. To our knowledge such an Ansatz
has never been attempted before. 

\begin{itemize}
    \item R.W. Hockney and J.W. Eastwood. {\it Computer Simulation Using Particles}. Adam Hilger, Philadelphia, 1988
    \item \url{http://en.wikipedia.org/wiki/P3M}
    \item \url{http://arxiv.org/abs/astro-ph/9805096}
    \item \url{http://arxiv.org/abs/astro-ph/0512030}
    \item maybe: \url{http://arxiv.org/pdf/cond-mat/9807099}
    \item maybe: \url{http://pdfserve.informaworld.com/927627_751307179_756919623.pdf} and references herein 
    \item check ada: \url{http://www2.fz-juelich.de/jsc/datapool/page/1907/p3mg.pdf} {\tt doi:10.1016/j.cpc.2005.03.077}
\end{itemize}

\section{Test Program}
There is a program {\tt test-update-1.cpp} the exercises all important parts of a particle tracking program. The grogram can be found at 
{\tt \$IPPL\_ROOT/test/particle}. This program should be used to perform all test.


\section{Timetable}

In Table \ref{tbl:time} we propose a timetable.

%\TODO[FINISH]{Timetable}
\begin{table}[ht]\footnotesize
\begin{center}  
\begin{tabular}{lclllll} 
\hline 
\bf & \bf Week  & \bf Activity & \bf Milestone  \\
\hline \\
%\hline
1 & 1-2 & review update/repart/bound  & gain understanding of underlaying \\
  &     & and draw flow diagrams      & mechanisms of these important IPPL methods. \\
2 & 3-4 & re-implement update/repart/bound  & new update/repart/bound \\
  &     & validate and benchmark      &  remove obstacles described in 2.1, \ippl~manual update\\
3 & 5-7 & test \texttt{CachedParticleLayout}  &  \texttt{CachedParticleLayout} tested and benchmarked \\
  &     & and maybe fix issues     &  write chapter in \ippl~manual \&\\
4 & 8-12 & implement  P$^3$M & start writeup for a paper \\
  &     &      &  \\
\\
\hline 
\end{tabular} 
\caption{Time schedule of the proposed project}
\label{tbl:time}
\end{center}
\end{table}

\end{document}
