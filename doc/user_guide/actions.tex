
%=================================================
%=================================================
%
% WARNING: NOT USED IN opal_user_guide.tex
%
%=================================================
%=================================================

\chapter{Action Commands}
\label{chp:action}
\index{Action|(}

\section{Normal-Form Analysis}
\index{Normal Form!Dynamic}
\index{Normal Form!Static}
\index{Dynamic!Normal Form}
\index{Static!Normal Form}
Please note Tables are not yet supported in \noopalt and \noopalcycl. Normal-Form Analysis does not make sense in a non map based computation.
The two commands
\begin{verbatim}
DYNAMIC, LINE=name, BEAM=name, FILE=string, ORDER=integer;
STATIC, LINE=name, BEAM=name, FILE=string, ORDER=integer;
\end{verbatim}
both evaluate the truncated Taylor series map for one turn up to order
\texttt{ORDER} and perform normal-form analysis on the result.
Both have the following attributes:
\begin{kdescription}
\item[LINE]
The label of a \textbf{beam line or sequence} \seechp{lines} defined
previously (no default).  Its transfer map will be evaluated and
analysed to the desired order.
\item[BEAM]
The label of a \keyword{BEAM}~command \seechp{beam} defined
previously (default = \texttt{UNNAMED\_BEAM}).
\index{UNNAMED\_BEAM}
It defines the charge, kinetic energy, rest mass,
and reference velocity for the reference
particle.
\item[FILE]
The name of the file to be written
(default = "\texttt{dynamic}" or "\texttt{static}" respectively).
\item[ORDER]
The maximum order for the map.
\end{kdescription}

\subsection{Normal-Form Analysis for Dynamic Map}
\label{sec:dynamic}
\index{DYNAMIC}
The \keyword{DYNAMIC} command interprets the transfer map as dynamic,
i.e. there is synchrotron motion with an average momentum $p_s$.

\subsection{Normal-Form Analysis for Static Map}
\label{sec:static}
\index{STATIC}
The command assumes that there are no cavities,
and interprets the transfer map as static,
i.e. the particles have constant momentum.
The \keyword{STATIC} command also finds the fixed point of the map,
that is the variation of the closed orbit with momentum.
\index{Action|)}