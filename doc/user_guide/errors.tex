%=================================================
%=================================================
%
% WARNING: NOT USED IN opal_user_guide.tex
%
%=================================================
%=================================================

\chapter{Error Definitions}
\label{chp:error}
\index{Imperfections}
\index{Errors|(}
Please note Error Definitions are not yet supported in \noopalt and \noopalcycl.
\section{Error Commands}
\label{sec:errorcmd}

\begin{table}[ht] \footnotesize
  \begin{center}
    \caption{Error Commands}
    \label{tab:errorcmd}
    \begin{tabular}{|p{0.3\textwidth}|p{0.6\textwidth}|}
      \hline
      \tabhead{Command & Purpose}
      \hline
      \tabline[sec:variable]{name=expression}{Parameter relation}
      \tabline[sec:errormode]{ERROR}{Start error mode}
      \tabline[sec:select]{SELECT}{Select elements to be affected}
      \tabline[sec:erroralign]{EALIGN}{Define misalignment errors}
      \tabline[sec:errorfield]{EFIELD}{Define multipole field errors (by amplitude and rotation)}
      \tabline[sec:errorfield]{EFCOMP}{Define multipole field errors (by components)}
      \tabline[sec:errorprint]{EPRINT}{Print error definitions}
      \tabline[sec:errorsave]{ESAVE}{Save error definitions to file}
      \tabline[sec:errormode]{ENDERROR}{End error mode}
      \hline
    \end{tabular}
  \end{center}
\end{table}

\section{Error Mode}
\label{sec:errormode}
\index{ERROR}
\index{ENDERROR}
\index{Error Mode}
\index{Mode!Error}

For all error-defining commands \seetab{errorcmd},
the beam line \seesec{line} or sequence \seesec{sequence} to be
affected must be specified with the command
\begin{verbatim}
ERROR, LINE=name, SEED=real, ADD=logical, CLEAR=logical;
\end{verbatim}
It has the following attributes:
\begin{kdescription}
\item[LINE]
  The line to be affected.
\item[SEED]
  Selects a particular sequence of random values.
  A SEED value is an integer in the range [0...999999999] (default: 123456789).
  SEED can be an expression.
  See also: random values, \secref{adefer}.
\item[ADD]
  If this value is entered as true, the subsequent error definitions
  are added to any existing ones.
  Otherwise, the subsequent error definitions overwrite any existing
  ones.
\item[CLEAR]
  If this value is true, all errors are erased from the given line.
\end{kdescription}
After this command \opal enters ``error mode'', in which it accepts only
the error defining commands \seetab{errorcmd}.
Normal mode is resumed by entering the command
\begin{verbatim}
ENDERROR;
\end{verbatim}
It is possible to assign alignment errors and field errors
to single beam elements or to ranges of beam elements.
Errors can be entered both with given or with random values.

\section{Define Misalignments}
\label{sec:erroralign}
\index{EALIGN}
\index{Error!Alignment}
\index{Misalignment}
Alignment errors are defined by the \texttt{EALIGN} command.
The misalignments refer to a local reference system \seefig{local}
for a perfectly aligned machine, placed half-way through the element
on the design orbit.
Possible misalignments are displacements along the three coordinate axes,
and rotations about the coordinate axes.
Alignment errors can be assigned to all beam elements except drift spaces.
The effect of misalignments is treated in a linear approximation.

Beam position monitors \seesec{monitor}
can be given read errors in both horizontal and vertical planes.
Monitor read errors are ignored for all other elements.
Each new \texttt{EALIGN} statement replaces the misalignment errors
for all elements in its range, unless \texttt{ERROR,ADD}
has been entered.

The elements to be misaligned must first be selected by the
{SELECT} command \seesec{select}.
Alignment errors can then be defined by the statement
\begin{verbatim}
EALIGN, DX=real, DY=real, DS=real, DPHI=real,
        DTHETA=real,DPSI=real, MREX=real,MREY=real;
\end{verbatim}
Its attributes are:
\begin{kdescription}
\item[DX]
  The misalignment in the $x$-direction
  for the entry of the beam element (default: 0 m).
  A positive value displaces the element
  in the positive $x$-direction \seefig{xsdisp}.
\item[DY]
  The misalignment in the $y$-direction
  for the entry of the beam element (default: 0 m).
  A positive value displaces the element
  in the positive $y$-direction \seefig{ysdisp}.
\item[DS]
  The misalignment in the $s$-direction
  for the entry of the beam element (default: 0 m).
  A positive value displaces the element
  in the positive $s$-direction \seefig{xsdisp}.
\item[DPHI]
  The rotation around the $x$-axis, (default: 0 rad).
  A positive angle \seefig{xsdisp} gives a greater $x$-coordinate for the exit
  than for the entry.
\item[DTHETA]
  The rotation around the $y$-axis (default: 0 rad) according to
  the right hand rule \seefig{xsdisp}.
\item[DPSI]
  The rotation around the $s$-axis (default: 0 rad) according to
  the right hand rule \seefig{xydisp}.
\item[MREX]
  The horizontal read error for a beam position monitor (default: 0 m).
  This is ignored if the element is not a monitor
  For a positive value \seefig{monitor} the reading for $x$ is too high.
\item[MREY]
  The vertical read error for a beam position monitor (default: 0 m).
  This is ignored if the element is not a monitor.
  For a positive value \seefig{monitor} the reading for $y$ is too high.
\end{kdescription}
Example:
\begin{verbatim}
EALIGN, QF[2], DX=0.002, DY=0.0004*RANF(),
        DPHI=0.0002*GAUSS();
\end{verbatim}
See also:
\begin{itemize}
\item Deferred Expressions \seesec{adefer},
\item Random Generators \seesec{adefer}.
\end{itemize}

\begin{figure}[ht]
\begin{center}
\begin{picture}(400,140)(0,30)
\put(50,75){\line(1,0){96}}
\put(154,75){\vector(1,0){246}}
\put(50,75){\makebox(0,0)[r]{\shortstack{original \\beam line}}}
\put(390,65){\makebox(0,0)[r]{s}}
\put(150,75){\circle{8}}
\put(150,75){\circle*{2}}
\put(140,85){\makebox(0,0){y}}
\put(150,30){\line(0,1){41}}
\put(150,79){\vector(0,1){91}}
\put(140,160){\makebox(0,0){x}}
\put(130,55){\vector(1,1){15}}
\put(130,55){\makebox(0,0)[tr]{\shortstack{original entrance \\
of the magnet}}}
\put(150,55){\vector(1,0){50}}
\put(175,45){\makebox(0,0){DS}}
\put(130,75){\vector(0,1){50}}
\put(125,100){\makebox(0,0)[r]{DX}}
\put(125,125){\line(1,0){185}}
\put(200,50){\line(0,1){120}}
\put(200,125){\circle*{4}}
\put(168,117){\vector(4,1){144}}
\put(210,85){\vector(-1,4){20}}
\bezier{30}(300,125)(300,137)(297,149)
\put(299,140){\vector(-1,4){2}}
\put(305,137){\makebox(0,0)[l]{DTHETA}}
\put(282,137){\line(-1,4){4}}
\put(202,117){\line(-1,4){4}}
\put(198,133){\line(4,1){80}}
\put(202,117){\line(4,1){80}}
\put(150,67){\line(1,0){80}}
\put(150,83){\line(1,0){80}}
\put(150,67){\line(0,1){4}}
\put(150,79){\line(0,1){4}}
\put(230,67){\line(0,1){16}}
\end{picture}
\caption{Example of Misplacement in the $(x, s)$-plane.}
\label{fig:xsdisp}
\end{center}
\end{figure}

\begin{figure}[ht]
\begin{center}
\begin{picture}(400,140)(0,30)
\put(50,75){\line(1,0){96}}
\put(154,75){\vector(1,0){246}}
\put(50,75){\makebox(0,0)[r]{\shortstack{original \\beam line}}}
\put(390,65){\makebox(0,0)[r]{s}}
\put(150,75){\circle{8}}\put(150,75){\makebox(0,0){$\times$}}
\put(140,85){\makebox(0,0){x}}
\put(150,30){\line(0,1){41}}
\put(150,79){\vector(0,1){91}}
\put(140,160){\makebox(0,0){y}}
\put(130,55){\vector(1,1){15}}
\put(130,55){\makebox(0,0)[tr]{\shortstack{original entrance \\
of the magnet}}}
\put(150,55){\vector(1,0){50}}
\put(175,45){\makebox(0,0){DS}}
\put(130,75){\vector(0,1){50}}
\put(125,100){\makebox(0,0)[r]{DY}}
\put(125,125){\line(1,0){185}}
\put(200,50){\line(0,1){120}}
\put(200,125){\circle*{4}}
\put(168,133){\vector(4,-1){144}}
\put(189,81){\vector(1,4){22}}
\bezier{30}(300,125)(300,113)(297,101)
\put(299,109){\vector(-1,-4){2}}
\put(305,113){\makebox(0,0)[l]{DPHI}}
\put(278,97){\line(1,4){4}}
\put(198,117){\line(1,4){4}}
\put(202,133){\line(4,-1){80}}
\put(198,117){\line(4,-1){80}}
\put(150,67){\line(1,0){80}}
\put(150,83){\line(1,0){80}}
\put(150,67){\line(0,1){4}}
\put(150,79){\line(0,1){4}}
\put(230,67){\line(0,1){16}}
\end{picture}
\caption{Example of Misplacement in the $(y, s)$-plane.}
\label{fig:ysdisp}
\end{center}
\end{figure}

\begin{figure}[ht]
\begin{center}
\begin{picture}(400,150)
\put(204,75){\line(1,0){96}}
\put(196,75){\vector(-1,0){96}}
\put(305,75){\makebox(0,0)[l]{\shortstack{horizontal \\plane}}}
\put(110,65){\makebox(0,0)[l]{x}}
\put(200,75){\circle{8}}
\put(200,75){\makebox(0,0){\(\times\)}}
\put(210,65){\makebox(0,0){s}}
\put(200,0){\line(0,1){71}}
\put(200,79){\vector(0,1){71}}
\put(190,140){\makebox(0,0){y}}
\put(196,76.3){\vector(-3,1){70}}
\put(204,73.7){\line(3,-1){70}}
\put(198.7,72){\line(-1,-3){23.3}}
\put(201.3,78){\vector(1,3){23.3}}
\bezier{20}(260,75)(260,65)(257,56)
\put(259,62){\vector(-1,-3){2}}
\put(265,67){\makebox(0,0)[l]{DPSI}}
\put(160,55){\line(3,-1){60}}
\put(180,115){\line(3,-1){60}}
\put(160,55){\line(1,3){20}}
\put(220,35){\line(1,3){20}}
\put(202.8,77.8){\line(1,1){50}}
\put(197.2,72.2){\line(-1,-1){50}}
\bezier{20}(213,114)(222,111)(229,104)
\put(227,106){\vector(1,-1){2}}
\put(233,125){\makebox(0,0){ROT}}
\end{picture}
\caption{Example of Misplacement in the $(x, y)$-plane.}
\label{fig:xydisp}
\end{center}
\end{figure}

\begin{figure}[ht]
\begin{center}
\begin{picture}(400,150)(0,20)
\put(204,75){\line(1,0){96}}
\put(196,75){\vector(-1,0){196}}
\put(305,75){\makebox(0,0)[l]{\shortstack{horizontal \\plane}}}
\put(10,65){\makebox(0,0)[l]{x}}
\put(200,75){\circle{8}}\put(200,75){\makebox(0,0){\(\times\)}}
\put(220,55){\vector(-1,1){15}}
\put(220,55){\makebox(0,0)[tl]{\shortstack{true beam \\position}}}
\put(210,85){\makebox(0,0){s}}
\put(200,30){\line(0,1){41}}
\put(200,79){\vector(0,1){101}}
\put(190,170){\makebox(0,0){y}}
\put(100,50){\vector(0,1){125}}
\put(150,125){\vector(-1,0){100}}
\put(100,125){\circle*{4}}
\put(80,145){\vector(1,-1){15}}
\put(80,145){\makebox(0,0)[br]{\shortstack{beam position given \\
by the monitor}}}
\put(200,55){\vector(-1,0){100}}
\put(150,45){\makebox(0,0){MREX}}
\put(80,75){\vector(0,1){50}}
\put(70,100){\makebox(0,0)[r]{MREY}}
\end{picture}
\caption{Example of Read Errors in a monitor.}
\label{fig:monitor}
\end{center}
\end{figure}

\section{Define Field Errors}
\label{sec:errorfield}
\index{EFCOMP}
\index{EFIELD}
\index{Error!Field}
\index{Field!Error}

Field errors can be entered as relative or absolute errors.
Different multipole components can be specified with
different kinds of errors (relative or absolute).
If an attempt is made to assign both a relative
and an absolute error to the same multipole component,
the absolute error is used and a warning is given.
Relations between absolute and relative field errors are listed below.

All field errors are specified as the integrated value $\int K_k ds$ of the
field components along the magnet axis in $\mathrm{m}^{-k}$.
At present field errors may only affect field components
allowed as normal components in a magnet.
This means for example that a dipole may have errors of the type dipole,
quadrupole, sextupole, and octupole; but not of the type decapole.
There is no provision to specify a global relative
excitation error affecting all field components in a combined function
magnet.
Such an error may only be entered by defining the same relative error
for all field components.

First the elements to be affected must be selected by the
{SELECT} command \seesec{select}.
Field errors can then be specified for magnetic elements
by one of the statements
\begin{verbatim}
EFIELD, ORDER=integer, RADIUS=real, ROT=deferred-vector,
        DKR=deferred-vector, DK=deferred-vector;
EFCOMP, ORDER=integer, RADIUS=real,
        DKN=deferred-vector, DKS=deferred-vector,
        DKNR=deferred-vector, DKSR=deferred-vector;
\end{verbatim}
Each new \texttt{EFIELD} or \texttt{EFCOMP} statement
replaces the field errors for all selected elements.
Any old field errors present in the range are discarded
or incremented depending on the setting of \texttt{ERROR,ADD}.
\texttt{EFIELD} defines the error in terms
of relative or absolute amplitude and rotation;
while \texttt{EFCOMP} defines them in terms
of relative or absolute components.

Both commands have the attributes:
\begin{kdescription}
\item[ORDER]
  If relative errors are entered for multipoles,
  this defines the order of the base component to which the relative
  errors refer.
  The default is zero.
\item[RADIUS]
  Radius $R$ were the relative components are specified (default 1 m).
  This attribute is required if any relative component is used.
\end{kdescription}
The command \texttt{EFIELD} has the following additional attributes:
\begin{kdescription}
\item[DK]
  A {vector} \seesec{vector} of deferred expressions \seesec{adefer}.
  Its component $k$ is the absolute error amplitude
  for the multipole strength with $(2k+2)$-poles
  (default: $0 \mathrm{m}^k$).
\item[DKR]
  A {vector} \seesec{vector} of deferred expressions \seesec{adefer}.
  Its component $k$ is the relative error amplitude
  for the multipole strength with $(2k+2)$-poles
  (default: $0 \mathrm{m}^k$).
\item[ROT]
  A {vector} \seesec{vector} of deferred expressions \seesec{adefer}.
  Its component $k$ is the rotation angle \seefig{xydisp}
  for the multipole strength with $(2k+2)$-poles
  (default: 0 rad).
\end{kdescription}
The command \texttt{EFCOMP} has the following additional attributes:
\begin{kdescription}
\item[DKN]
  A {vector} \seesec{vector} of deferred expressions \seesec{adefer}.
  Its component $k$ is the absolute error for the normal multipole
  strength with $(2k+2)$-poles (default: $0 \mathrm{m}^k$).
\item[DKS]
  A {vector} \seesec{vector} of deferred expressions \seesec{adefer}.
  Its component $k$ is the absolute error for the skew multipole
  strength with $(2k+2)$-poles (default: $0 \mathrm{m}^k$).
\item[DKNR]
  A {vector} \seesec{vector} of deferred expressions \seesec{adefer}.
  Its component $k$ is the relative error for the normal multipole
  strength with $(2k+2)$-poles (default: $0 \mathrm{m}^k$).
\item[DKSR]
  A {vector} \seesec{vector} of deferred expressions \seesec{adefer}.
  Its component $k$ is the relative error for the skew multipole
  strength with $(2k+2)$-poles (default: $0 \mathrm{m}^k$).
\end{kdescription}
All random vectors have an unlimited length.

Examples:
\begin{verbatim}
EFIELD, DK={0, 0.0025*RANF(), 0, 0, 0,
            DK(5)=0.0092*GAUSS()};
EFIELD, DKR={0, 0, 0, 0.0025*RANF(), 0,
             DKR(5)=0.0092*GAUSS()};
\end{verbatim}
See also:
\begin{itemize}
\item Multipole Field Errors \seesec{errorfield},
\item Sign Conventions for magnetic fields \seesec{sign},
\item Attribute Vectors \seesec{vector}
\item Deferred Expressions \seesec{adefer},
\item Random Generators \seesec{adefer}.
\end{itemize}

\subsection{Multipole Field Errors}
\label{sec:errormulti}
Multipole field errors can be assigned to all multipole-like elements:
\begin{itemize}
\item \texttt{RBEND} \seessec{RBend},
\item \texttt{SBEND} \seessec{SBend},
\item \texttt{QUADRUPOLE} \seesec{quadrupole},
\item \texttt{SEXTUPOLE} \seesec{sextupole},
\item \texttt{OCTUPOLE} \seesec{octupole},
\item \texttt{MULTIPOLE} \seesec{multipole},
\end{itemize}
The main field and the error field are evaluated separately and then added.
Error field components can be absolute or relative values.
Relative errors are referred to the amplitude of the nominal component
of order \texttt{ORDER} = $n$ (default = 0),
taken at the radius \texttt{RADIUS} = $R$ (default = 1).

Given the factors
\[
\psi_k = (k + 1) * \texttt{ROT[k]},
c_k = \cos \psi_k,
s_k = - \sin \psi_k,
f_k = \texttt{RADIUS[k]}^(k-n) * n! / k!.
\]
the field components
$\delta K_{kn}$ (normal) and $\delta K_{ks}$ (skew)
are computed as follows from the input data:
\[
\begin{array}{lclclclcl}
  \delta K_{kn} &=& c_k \mathtt{DK[k]} &=& f_k c_k \mathtt{DKR[k]} &=&
  \mathtt{DKN[k]} &=& f_k \mathtt{DKNR[k]} \\
  \delta K_{ks} &=& s_k \mathtt{DK[k]} &=& f_k s_k \mathtt{DKR[k]} &=&
  \mathtt{DKS[k]} &=& f_k \mathtt{DKSR[k]}
\end{array}
\]
The first line gives the normal components,
the second line the skew components.

\section{Print Machine Imperfections}
\label{sec:errorprint}
\index{EPRINT}
\index{Error!Print}
\index{Print!Errors}

Before imperfections can be listed,
the elements to be printed must be selected by a
{SELECT} command \seesec{select}.
Then a table of errors assigned to these elements can be printed
by the command
\begin{verbatim}
EPRINT, FILE=string, ALIGN=logical, FIELD=logical,
        ORDER=integer, RADIUS=real;
\end{verbatim}
The command has the following attributes:
\begin{kdescription}
\item[FILE]
  The file to receive the output.
\item[ALIGN]
  If true, the alignment errors are printed.
\item[FIELD]
  If true, the field errors are printed.
\item[ORDER]
  The order used for normalizing, like on \texttt{EFIELD} and
  \texttt{EFCOMP}.
\item[RADIUS]
  The radius used for normalizing, like on \texttt{EFIELD} and
  \texttt{EFCOMP}.
\end{kdescription}

\section{Save Machine Imperfections}
\label{sec:errorsave}
\index{ESAVE}
\index{Error!Save}
\index{Saving!Errors}

Before imperfections can be saved,
the elements to be saved must be selected by a
{SELECT} command \seesec{select}.
The element imperfections can then be saved to a file with the command
\begin{verbatim}
ESAVE, FILE=string, ALIGN=logical, FIELD=logical,
       ORDER=integer, RADIUS=real;
\end{verbatim}
This command saves a table of errors assigned to elements on a file,
using a format which can be read again by \opal to obtain the same results.
This allows dumping the errors and reloading them for the same line.
Field errors are saved as relative errors,
and the \texttt{RADIUS} and \texttt{ORDER} must be entered as used on the
{EFIELD} \seesec{errorfield} or
{EFCOMP} \seesec{errorfield} commands.
%If {OPTION,\opal8} \seesec{option} is active,
%the output is given in \opal-8 format.
The command has the following attributes:
\begin{kdescription}
\item[FILE]
  The file to receive the output.
\item[ALIGN]
  If true, the alignment errors are saved.
\item[FIELD]
  If true, the field errors are saved.
\item[ORDER]
  The order used for normalizing, like on \texttt{EFIELD} and
  \texttt{EFCOMP}.
\item[RADIUS]
  The radius used for normalizing, like on \texttt{EFIELD} and
  \texttt{EFCOMP}.
\end{kdescription}

\index{Errors|)}