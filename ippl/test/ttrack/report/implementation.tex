\pagebreak
\section{Implementation}
The current implementation of the Touschek Lifetime into a program is called ttrack (Touschek tracker). It is written in C/C++ and is based on $IP^2L$, and is to designed to run fully parallel. You to specify linear lattices and beam geometries in the input file.
\subsection{Scaling}
As the following figure shows the program, doesn't scale very well, as the processor number gets higher. However, this might be due to the particle number being very low. We used $10^5$ particles in the simulation. This means that there are less than 400 particles per processor in average.
\begin{figure}[here]
\centering
 \includegraphics[width=0.80\textwidth]{scaling.pdf}
 \caption{Scaling of ttrack}
\end{figure}
