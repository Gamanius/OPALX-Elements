\section{Direct Monte Carlo Simulation}
This approach has failed. However, it is still presented here to give people who want to try it again, about what to do and what not to do.
\subsection{Coulomb Force}
The coulomb force works between 2 charges, here 2 electrons. It is given by:
\begin{equation} \vec F = \frac {q^2} {4 \pi \epsilon_0} \cdot \frac {\vec r_1 - \vec r_2}{|\vec r_1 - \vec r_2|^3}. \end{equation}
To compute this force we have to transform to the beam frame. In the laboratory frame we would also have to do with magnetic fields. This mainly means, that the longitudinal component goes into the calculations with a factor of $\gamma$.

\subsection{Electron - Electron Scattering}
For the electron - electron scattering in a plane, I am using the formulas given by [6]. I assume that we are having strong and complete collisions. So I get for the longitudinal velocity shift after (6.10.4):
\begin{equation} \Delta v_z \approx \frac{m v^3 b_z}{2 C_0 (1 + q_c)} \end{equation}
where $C_0 = q^2 / 4 \pi \epsilon_0$ and $q_c = (m b v^2/2 C_0)^2 = (m/2C_0 \cdot b v^2)^2 \approx 10^8$ which means that the last equation can further be approximated by $1 + q_c \approx q_c$ to:
\begin{equation} \Delta v_z \approx  \frac{2 C_0} m \cdot \frac 1 v \frac {b_z}{b^2}. \end{equation}
Taking into account, that $\Delta t = \Delta s/v$, one sees that this is about to what is given by the Coulomb force except for the factor of 2. The first factor has a value of $2 C_0 / m \approx 506 m^3 s^{-2}$.\\
This formula also arises in [9] where he assumes, that the force can be calculated from the undisturbed trajectory and the force at point where the impact parameter is smallest.\\
For the transversal velocity shift one has after (6.11.3):
\begin{equation} \Delta v_\perp \approx \frac v {1 + q_c}\left[ 1 + q_c \left(\frac{r_\perp \sin(\Phi)^2}{b}\right)^2\right]^{1/2}. \end{equation}

\subsubsection{Adapted to 3D}
At the beginning of the scattering we have the relative distance and velocity:
\begin{eqnarray} \vec r &=& \vec r_{field} - \vec r_{test} \\
\vec v &=& \vec v_{field} - \vec v_{test} .\end{eqnarray}
From these we derive our coordinate system as done in (6.3.1) in [6]:
\begin{eqnarray}
\hat a &=& - \frac{\vec v}{|\vec v|} \\
\hat b &=& \frac{\vec r - (\vec r \cdot \hat a) \hat a}{|\vec r - (\vec r \cdot \hat a) \hat a|}.
\end{eqnarray}
We only have a 2 dimensional system, since the angular momentum is conserved, and the motion of 2 particles then takes place in a plane.\\
In this coordinate system the relative changes of the velocity are given by (6.7.13) having already done the limes $T \rightarrow \infty$:
\begin{eqnarray}
\Delta \dot a &\approx& \frac{2v}{1 + q_c} \\
\Delta \dot b &\approx& \frac{2 \sqrt{q_c}}{1 + q_c} \cdot v
\end{eqnarray}
leading to the velocity change of one electron using (6.10.1):
\begin{equation} \Delta v = - \frac{\Delta \dot a \hat a + \Delta \dot b \hat b} 2. \end{equation}
The factor 1/2 comes from passing from 1 particle in a central force field to 2 identical particles in each others force field. For more details on the derivation see [6].
\subsubsection{When does this work?}
Deriving the formulas above makes use of approximations, so you need to ask yourself if you are allowed to do it. On page 152 of [6] the following condition is given:
\begin{equation} \frac r {r_p} \gg 1.5 \end{equation}
r being the radius at the moment of the beginning of the collision. $r_p$ being the perihelion distance, the smallest distance from the particle to the center of scattering. How to compute $r_p$ is shown below.\\
The perihelion distance $r_p$  can be computed by:
\begin{eqnarray}
\vec J &=& \frac 1 2 m \vec r \times \vec v \\
E &=& \frac 1 4 m v^2 + \frac{C_0} r \\
q &=& \frac{ 4 J^2 E}{C_0^2 m} \\
\epsilon &=& (1 + q)^{1/2} \\
r_p &=& \frac J {\sqrt{m E}} \sqrt{\frac{\epsilon + 1}{\epsilon -1}}.
\end{eqnarray}
Checking this in the program, showed that we are not allowed to use this approach. However, this approach also has the problem, that it assumes non-relativistic motion of the electrons, which is not the case.

\subsubsection{Simulation Consideration}
If you look at these formulas carefully, you will see that the effect is biggest if $v_z$ is small, or you find that out by trying around with your program as I did. You see this by that $\Delta \dot b = \sqrt{q_c} \Delta \dot a$, $\sqrt{q_c}$ is a big number, and $\hat a$ points in the direction of v and $\hat b$ points orthogonal to it. Since we are interested in big changes in the longitudinal direction, we want $\hat a$ to have a small part in the longitudinal direction, so $v_z$ has to be small.

\subsection{Idea behind the Direct Monte Carlo Simulation}
The model for simulation did the following. The simulation of the electrons going around the storage ring is done by linear beam optics. Then by each time step, we compute the list of electrons, which are in an interaction sphere of each other.\\
The radius of this sphere was given by a characteristic length of the interactions. In our case this was the Debye Length.\\
If this sphere is none empty, we collide the electron in the center with one electron out of this sphere. After 2 electrons have collided, they are marked, so the program doesn't collide them a second time. So each electron only collides once on each time step.	
\subsubsection{Overview}
\begin{enumerate}
\item Compute effect of the storage ring using linear beam optics.
\item Determine the list of possible interaction partners, if particles are not marked off.
\item Pick out the closest interaction partner.
\item Do interaction, we have the option what to do here.
\item Mark particle off.
\item Do this for each particle.
\item Start again at 1.
\end{enumerate}

\subsubsection{Difference to plasma simulation}
When people are simulating plasmas as [7] or [8], they usually chose interaction partners at random inside a plasma cell. This is exactly opposite to what we are doing with always picking out the nearest neighbor. However we are also interested into something different. When simulating plasmas, people are interested in the overall behavior of the plasma. On the other hand we are interested in the behavior of particles under collisions, when they come very close to each other. So we have to look for these events. At least I am thinking this is the right way to do it. 

\subsubsection{Time step}
In order to simulate our dynamic system, we need a discrete time step to advance the system. This time step should be big, so we can achieve long simulation times. However, we also have to make it small to reduce simulation mistakes. We already have a characteristic length in the simulation: the Debye length. So we could for example pick our time step so, that a reference particle, only can get through half of a Debye sphere of another electron.
\subsubsection{Picking out the interaction partner}
First we wanted to pick out the interaction partner at random in the Debye sphere. However this lead to a huge depency of the number of Touschek Loss events to the interaction radius. So this wasn't a good way. So now we are always picking out the closest neighbor inside the Debye sphere.

\subsection{Scaling}
Since we cannot simulate each particle in the beam, we have to use some kind of macro particles. This means that each simulated particles represents several real particles. So we have to take care of this in our simulation. If we simulate $N_{sim}$ Particles, and in reality there would be $N_{real}$ particles, we have a factor:
\begin{equation} F = \frac{N_{real}}{N_{sim}}. \end{equation}
If we consider the distance between 2 particles. It becomes smaller, when we are adding more particles. Since our simulation is taking place in 3D, we would have:
\begin{equation} r \rightarrow \frac r {F^{1/3}}. \end{equation}
However we should also consider the number of particles lost, if one macro particle is lost.

\subsubsection{From simulation to reality}
Having acquired simulation data, we want to say something about the real world with it. To do this, we assume that we have only computed the fraction of the scatterings, that $Number_{sim}$ particles would have done in the beam. From this number we extrapolate to the real number of particles by:
\begin{equation} Loss_{real} = Loss_{sim} \cdot \frac{Number_{real}}{Number_{sim}} \end{equation}
This means that we assume, that we can represent the whole beam with fewer particles and scaling the distances.
\subsubsection{Computed Value}
In our simulation we are no computing $\alpha$, as:
\begin{equation} \alpha = \frac{\Delta N_{turn}}{\Delta t_{Turn}} \cdot \frac 1 {N_{sim} ^2} \cdot  \frac{N_{real}}{N_{sim}}. \end{equation}
We are using for the error computation:
\begin{equation} \alpha_{lower} = \alpha \cdot (1 - \frac 1 {\sqrt{\Delta N_{total}}}), \  \alpha_{upper} = \alpha \cdot (1 + \frac 1 {\sqrt{\Delta N_{total}}}). \end{equation}

\subsection{Dependence on parameters}
In order to check if our simulation works well, we can test if it scales well with the parameters going into the model, like the bunch charge, $\gamma$, the momentum acceptance and the bunch shape.\\
We just got the scaling with $\gamma$ to work right. An independence on the time step could be achieved by passing from $\Delta t$ the length of a time step, to something depending on an invariant distance and the velocity of the particle an option is $\tau = b / v$, where b is the impact parameter. However, we never tried to run this with correct relativistic mechanics.
