\chapter{Centering}
\label{sec:cent}
Classes in this chapter represent the abstraction of centering at particular positions on a \texttt{Mesh}. 
For any type of mesh, vertex (mesh-node) centering is well-defined. For Cartesian meshes, cell centering is well defined, because each 
rectangular mesh cell has a well-defined center, the midpoint along each dimension. Also, for Cartesian meshes, combinations of cell and vertex centering 
along the various dimensions clearly define common centerings such as face centering and edge centering. \ippl predefines classes for cell and vertex centering, 
as well as classes to represent all possible combination centerings for Cartesian meshes. These classes are all static in the current implementation 
(you do not instantiate objects of the class types, you just refer to the centerings by class name). The centering classes typically serve as the centering 
template parameter for Field and other \ippl classes parameterized on centering. 
\section{\texttt{Cell} Class}
The static \texttt{Cell} class represents the abstraction of cell-centering of something on a \texttt{Mesh}. 
That "something" is usually a \texttt{Field}. \texttt{Cell} can serve as a value for the \texttt{Centering} template parameter of the \texttt{Field} class. Every \texttt{Field} element, whatever its type, 
is centered at the position of the corresponding cell center in the \texttt{Mesh} using the same index space for mesh cell centers as 
(or the \texttt{Field} elements. If the Field elements (\texttt{T} parameter for \texttt{Field} class) is some kind of multi component type such as \texttt{Vektor}, the whole object of that type in every \texttt{Field} 
element is cell-centered (that is, all components have the same centering). The \texttt{CartesianCentering} class allows you to center components of \texttt{Field} elements independently. 
If you center a \texttt{Field} on a \texttt{Mesh} using \texttt{Cell}, you must make sure that you construct the \texttt{Field} to have the proper number of elements so that it has one-for every cell in the mesh. 
When you use \texttt{Field} constructors taking a \texttt{Mesh} object argument (instance' of the class specified as the \texttt{Mesh} template parameter for \texttt{Field}), the extents of the 
\texttt{NDIndex} object or index field constructor arguments match those used to instantiate that \texttt{Mesh}. When you instantiate a \texttt{Field} with \texttt{Cell} centering using 
constructors without a mesh object argument, the internally-constructed mesh object will automatically have the right number of cells. 
subsection{Cell Definition (Public Interface)}
\begin{smallcode}
class Cell { 
public: 
    static char* CenteringName ;
    static void print_Centerings(ostream&) ;
}; 
\end{smallcode}



\subsection{\texttt{Cell} Constructors}
\texttt{Cell} is a static class you don't instantiate objects, but rather only refer to it by its class name. 

\subsection{\texttt{Cell} Member Functions and Member Data}

\begin{smallcode}
static char* CenteringName; 
\end{smallcode}
Public data member, having static value "\texttt{Cell}".
 
\begin{smallcode}
static void print_Centerings(ostream&) ;
\end{smallcode}
Invoked as \texttt{Cell::printCenterings(ostream\&)}  prints the value of the \texttt{string Cell::CenteringName}. 

\section{\texttt{Vert} Class}
The static \texttt{Vert} class represents the abstraction of vertex-centering of something on a mesh. 
That "something" is usually a \texttt{Field}. \texttt{Vert} can serve as a value for the \texttt{Centering} template parameter of the \texttt{Field} class. Every \texttt{Field} element, whatever its type, 
is centered at the position of the corresponding vertex center in the mesh-using the same index space for \texttt{Mesh} vertex centers as for the \texttt{Field} elements. If the \texttt{Field} elements 
(\texttt{T} parameter for \texttt{Field} class) is some kind of multicomponent type such as \texttt{Vektor}, the whole object of that type in every \texttt{Field} element is vertex-centered 
(that is, all components have the same centering). The \texttt{CartesianCentering} class allows you to center components of \texttt{Field} elements independently. 
If you center a \texttt{Field} on a \texttt{Mesh} using \texttt{Vert}, you must make sure that you construct the Field to have the proper number of elements so that it has one for every vertex in the mesh. 
When you use \texttt{Field} constructors taking a mesh object argument (instance of the class specified as the \texttt{Mesh} template parameter for \texttt{Field}), the extents of the \texttt{NDIndex} object or \texttt{Index} 
\texttt{Field} constructor arguments match those used to instantiate that \texttt{Mesh}. When you instantiate a \texttt{Field} with \texttt{Vert} centering using constructors without a mesh object argument, 
the internally constructed mesh object will automatically have the right number of vertices. 

subsection{Vert Definition (Public Interface)}
\begin{smallcode}
class Vert { 
public: 
    static char* CenteringName;
    static void print_Centerings(ostream&);
} ; 
\end{smallcode}


\subsection{\texttt{Vert} Constructors}
\texttt{Vert} is a static class you don't instantiate objects, but rather only refer to it by its class name. 

\subsection{\texttt{Vert} Member Functions and Member Data}

\begin{smallcode}
static char* CenteringName; 
\end{smallcode}
Public data member, having static value "\texttt{Vert}".
 
\begin{smallcode}
static void print_Centerings(ostream&) ;
\end{smallcode}
Invoked as \texttt{Vert::printCenterings(ostream\&)}  prints the value of the \texttt{string Vert::CenteringName}. 


\section{\texttt{CommonCartesianCenterings} Class}
The static CommonCartesianCenterings class is a wrapper class for commonly-used special cases ofthe \texttt{CartesianCentering} class, predefinedby \ippl as a convenience for the user. 
Basically, it is a collection of typedef's. \\
\texttt{CartesianCentering} is a parameterized static class representing the abstraction of componentwise centering of a multicomponent object (typically a \texttt{Field}) 
on a cartesian mesh. Via the template parameters, the user specifies the centering of the various components along the various directions. See next section for more details. 
\texttt{CommonCartesianCenterings} provides a, shorthand definition for some of the common cases expressible by \texttt{CartesianCentering}. \\
Also shown in the definition of \texttt{CommonCartesianCenterings} below are the \texttt{CenteringEnum} which it uses and the static wrapper class \\ \texttt{commonCartesianCenteringEnums}. 
This last class contains the special-case arrays of \texttt{CenteringEnum} values (\texttt{CELL} or \texttt{VERT}) which represent the various centerings for various specializations of the template parameters. \\
\texttt{CommonCartesianCenterings}  parameterized on dimensionality \texttt{Dim}, which has the same meaning as the dimensionality parameter in \texttt{Field} or \texttt{UniformCartesian} (for example). 

The unsigned value \texttt{NComponents} and the unsigned value \texttt{Direction}. \texttt{NComponents} represents the number of components in a multicomponent object to be centered on the mesh, 
for example, if you are centering a \texttt{Field<Vektor<double, 3U>} ,the number of components is three. Direction represents a specifying direction, and is only really used in some of the 
\texttt{CommonCartesianCenterings} members. For example, to specify face centering of a scalar field (or single component of a non-scalar field), you must specify which direction is perpendicular 
to the faces where you are centering. If you wanted to center a scalar field on the $xy$ faces in 3D, you would use the value \texttt{2U} for \texttt{Direction}; because the $xy$ faces are perpendicular to 
the $z$ direction (the directions are numbered sequentially, $x$ is \texttt{0U}, $y$ is \texttt{1U}, and $z$ is \texttt{2U}). 

\subsection{\texttt{CommonCartesianCenterings} Definition (Public Interface)}
\begin{smallcode}

enum CenteringEnum {CELL=0, VERTEX=1, VERT=1};
 
template<unsignedDim, unsigned NComponents=1U, unsigned Direction=0U> 
class CommonCartesianCenteringEnums
{ 
public: 
// CenteringEnum arrays Classes with simple, des'criptive names 

// All components of Field cell-centered in all directions: 
static CenteringEnum allCell [NComponents*Dim] ;
 
// All components of Field vertex-ceritered in all directions: 
static CenteringEnum allVertex[NComponents*Dim] ; 

// All components of Field face-centered in specified 
// direction (meaning vertex centered in that direction, cell-centered in others): 
static CenteringEnum allFace[NComponents*Dim];

// All components of Field edge-centered along specified direction
// (cell centered in that direction, vertex-centered in others): 
static CenteringEnum allEdge[NComponents*bim];, 

// Each vector component of Field face-centered in the corresponfing direction 
staticCenteringEnum vectorFace[NComponents*Dim]; 
} ; 

template<unsigned Dim, unsigned NComponents=1U, unsigned Direction=0U> 
class CommonCartesianCenterings 
{ 
public: 
typedef CartesianCentering<CommonCartesianCenteringEnums<Dim,NComponents,
Direction>::allCell, Dim, NComponents> allCell; 
typedef CartesianCentering<CommonCartesianCenteringEnums<Dim,NComponents,
Direction>::allVertex, Dim, NComponents> allVertex; 
typedef CartesianCentering<CommonCartesianCenteringEnums<Dim, NCo mponents,
Direction>::allFace, pim, NComponehts> allFace; 
typedef CartesianCentering<CommonCartesianCenteringEnums<Dim, NCo mponents,
Direction>::allEdge, Dim, NComponents> allEdge; 
typedef CartesianCentering<CommonCartesianCenteringEnums<Dim,NComponents,
Direction>::vectorFace, Dim, NComponents> vectorFace; 
}; 
\end{smallcode}


\subsection{\texttt{CommonCartesianCenterings} Constructors}
\texttt{CommonCartesianCenterings} is a static class, you don't instantiate objects, but rather only refer to it by its class name. 
Refer to its members his way also for example: \texttt{CartesianCenterings<3U, 1U>} for the member representing centering on a 3D mesh of a scalar 
( one-component) object. 

\subsection{\texttt{CommonCartesianCenterings} Member Data}
\begin{smallcode}
typedef CartesianCentering<CommonCartesianCenteringEnums<Dim,NComponents,
Direction>::allCell, Dim, NComponents> allCell ;
\end{smallcode}
Specifies \texttt{CELL} centering of all components along all dimensions. Functionally equivalent to the \texttt{Cell} centering class. 

\begin{smallcode}
typedef CartesianCentering<CommonCartesianCenteringEnums<Dim,NComponents,
Direcition>::allVertex, Dim, NComponents> allVertex ;
\end{smallcode}
Specifies \texttt{VERTEX} centering of all components along all dimensions. Functionally equivalent to the \texttt{Vert} centering class. 

\begin{smallcode}
typedef CartesianCentering<CommonCartesianCenteringEnums<Dim, NComponents, 
Direction>::allFace, Dim, NComponents> allFace ;
\end{smallcode}
Specifies centering of all components on the faces which are orthogonal to the specified direction (Direction=0 means x, 1 means y, 2 means z). 
That is, vertex-centering along direction \texttt{Direction}, and cell-centering along all other directions. 

\begin{smallcode}
typedef CartesianCentering<CommonCartesianCenteringEnums<Dim, NComponents,
Direction>::allEdge, Dim, NComponents> allEdge ;
\end{smallcode}
Specifies centering of all components on the edges which are parallel to the specified direction (Direction=0 means x, 1 means y, 2 means z). That is, cell-centering 
along direction Direction, and vertex-centering along all other directions. 

\begin{smallcode}
typedef CartesianCentering<CommonCartesianCenteringEnums<Dim, NComponents,
Direction>::vectorFace, Dim, NComponents> vectorFace ;
\end{smallcode}

Specifies componentwise face centering of the components of a \texttt{Vector}. Usually you use this for centering a \ippl \texttt{Field} whose elements are \texttt{Vektor}'s. 
For example,
\begin{smallcode}
Field<Vektor<double, 3>, 3,UniformCartesian<3>,CommonCartesianCenterings<3,3,3>>
\end{smallcode}

\section{\texttt{CartesianCentering} Class}
The static \texttt{CartesianCentering} class represents the abstraction of component-wise centering of a multicomponent object (typically a \texttt{Field}) on a cartesian mesh. Via the template parameters, 
the user specifies the centering of the various components along the various directions.
 
\subsection{CartesianCenteringDefinition(Public Interface)}
\begin{smallcode}
template<const CenteringEnum* CE, unsigned Dim, unsigned NComponents=1U>
class CartesianCentering
{
public:
static char* CenteringName; 
static void print_Centerings(ostream&) ;
};
\end{smallcode}
 
\subsection{\texttt{CartesianCentering} Constructors}
\texttt{CartesianCentering} is a static class, you don't instantiate objects, but rather only refer to it by its class name, with fully-specified template parameter values.
If you have defined a static array of type \texttt{CenteringEnumcalled myCEArray}, the identifier \texttt{CartesianCenterings<myCEArray,3U,1U>} refers to the static class 
representing centering on a 3D mesh of a scalar (one-component) object according to the centering specifiers in \texttt{myCEArray}. In this case, \texttt{myCEArray} would 
have to have three elements (each having the value \texttt{CELL} or \texttt{VERT}). 
In general, the number of elements in the array of \texttt{CenteringEnum} (the CE template parameter value) must equal the value of the \texttt{Dim} parameter multiplied by the value of the \texttt{NComponent} parameter. 
The ordering of the elements is so that the component indices vary fastest, and the dimensions indices vary the slowest (like a 2D C array dimensioned as \texttt{[Dim][NComponent]}). 
You must declare the \texttt{CenteringEnum} array you use as the value of the CE template parameter as static, and at global scope.

\subsection{\texttt{CartesianCentering} Member Functions and Member Data} 
All the member functions of \texttt{CartesianCentering} are of course static. Refer to them with the syntax \texttt{classname::membername}, where classname is, a name identifying aparticular \texttt{CartesianCentering} 
class specialization, as described in this Appendix. 

\begin{smallcode}
static char* CenteringName ;
\end{smallcode}
A string containing a identifying name for a particular template instance of\\  \texttt{CartesianCentering}. Currently, there is only a single name, so that the value of 
\texttt{CartesianCentering<CE,Dim,NComponent> }is the same for any values of the template parameters. 

\begin{smallcode}
static void print_Centerings(ostream&) ;
\end{smallcode}
Prints a formatted version of the \texttt{CartesianCentering} class. Specifically, it prints the values of the elements of the \texttt{CE} template parameter (as an array with \texttt{Dim*NComponents} elements).

























