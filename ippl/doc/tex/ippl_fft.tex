\chapter{FFT}
\label{sec:fft}

The FFT class provides an interface for performing parallel Fourier transforms of various types on \ippl \texttt{\texttt{Field}} objects. FFT is templated on the type of transform to perform (\texttt{CCTransform}, \texttt{RCTransform}, or \texttt{SineTransform}); the dimensionality \texttt{\texttt{Dim}} of the fields to be transformed, and the floating-point precision type (either \texttt{float} or \texttt{double}). It is capable of transforming along all dimensions of a \texttt{\texttt{Field}} or only specified dimensions, and it handles all of the data transposes required to make the Fourier transforms efficient automatically. The FFT constructor arguments vary slightly depending upon which type of transform you wish to perform. Generally speaking, you provide an {\tt NDIndex} object or objects which contain the domains of the input and/or output \texttt{Field}s for the Fourier transform, an optional array of bools of length \texttt{Dim} indicating which dimensions are to be transformed (default is all dimensions), and an optional bool indicating whether or not to compress the intermediate \texttt{Field}s needed to perform data transposes when they are not in use. The default value of this optional argument is false, but the user can set this argument to true if it is necessary to conserve memory. For a complex--to--complex Fourier transform, the input and output fields are of the same element type and are the same size, so only one domain argument is needed. So in the simple case of transforming all dimensions of a \texttt{Field} of type {\tt complex<double>}, we would construct the FFT object with the code
\begin{smallcode}
FFT<CCTransform,Dim,double> ccfft(domain); 
\end{smallcode}
where domain is an {\tt NDIndex<Dim>} describing the domain of complex \texttt{Field}s to be transformed with the FFT object.
A real--to--complex Fourier transform takes a field of real numbers and returns a field of complex numbers (or vice-versa for an inverse complex--to--real transform), so we require separate domain arguments describing each \texttt{Field} in the FFT constructor. From the theory of Fourier mode analysis, we know that a Fourier transform of $N$ real numbers will produce $N/2+ 1$ unique complex modes, with modes $0$ and $N/2$ being purely real. Some FFT routines take advantage of the fact that if you pack together the real parts of modes $0$ and $N/2$ as one complex number, you can store all the resulting mode information in the same space as required for the input (i.e., $N$ real numbers or $N/2$ complex numbers). Such a technique tends to cause confusion in multidimensional real--to--complex FFTs, since mode data must then be separated out afterwards. So we choose a format in which the $N/2+ 1$ complex modes are stored separately as complex numbers. Thus, when a real--to--complex transform is performed on a {\tt \texttt{Field}} of doubles, the resulting {\tt \texttt{Field}} of type {\tt complex<double>} will have an extent one greater than half the length of the input field along the first dimension to be transformed and the same length along all other dimensions. This conformance of domains is checked by the FFT constructor. We might construct an FFT object for real--to--complex transforms with the line
\begin{smallcode}
FFT<RCTransform,Dim,double> rcfft(rdomain,cdomain,tdim); 
\end{smallcode}
where {\tt rdomain} and {\tt cdomain} are the conforming domains for the {\tt real} and {\tt complex} fields and {\tt tdim} is an array of bools indicating whether or not to transform each dimension. Note that we assume the axes of the field are to be transformed along in the order indicated by the domain arguments for a forward FFT and in the reverse order for an inverse FFT. Each \texttt{Index} object inside the provided domain should refer to a particular axis of the input \texttt{Field}, and these axes are transformed along in order. A sine transform is a special type of Fourier transform in which only the sine (odd) modes are retained. This transform has a field of real numbers for both its input and output, and its effect is to keep only that portion of the data which exhibits odd parity (i.e., vanishes at the endpoints of the interval). Typically, one wishes to enforce odd parity along one or more dimensions of a field, and then perform a standard real-to-complex transform along remaining dimensions. Hence, we require that the user provide two arrays of bools in the constructor: the first to indicate along which dimensions to perform a sine transform, and the second to indicate all of the transform dimensions (both sine transforms and standard FFTs). For example,
\begin{smallcode}
FFT<SineTransform,Dim,double> sinefft(rdomain,cdomain,sinedim,tdim);
\end{smallcode}
 

constructs an FFT object for doing sine transforms along the dimensions indicated by sinedim and a standard real-to-complex FFT over the other dimensions included in tdim. Alternatively, such transforms could be achieved in two steps, doing the sine transforms and the standard FFTs separately. In this case, we might construct our sine transform FFT object with the code
\begin{smallcode}
FFT<SineTransform,Dim,double> sinefft2(rdomain,sinedim); 
\end{smallcode}


and then construct a second FFT object for handling the real-to-complex transform. Note that a sine transform FFT object which is doing only sine transforms requires only a single domain argument describing the real input and output \texttt{Field}s in its constructor.

Once the appropriate FFT object has been constructed, a Fourier transform of data is invoked using the transform member function. The normal arguments to this function are an integer value of + 1 or -1 to indicate the sign of the exponential used in the transform (i.e., the direction of the transform, forward or inverse), and the input and output \texttt{Field}s. For this "two-field" form of the transform function, there is also an optional argument of type bool, which indicates whether or not the input \texttt{Field} is considered to be constant by the transform function. The default value of this optional argument is false, which allows the transform routine to attempt to use the input \texttt{Field} as temporary storage and avoid doing an additional data transpose. You should set the value of this argument to true if you must preserve the contents of the input \texttt{Field} for later use. We would use our previously constructed FFT object for real--to--complex transforms to perform a forward FFT in the following manner:
\begin{smallcode}
rcfft.transform(+l,realField,complexField);
\end{smallcode}
 

The results of the transform are automatically normalized such that a forward transform followed by an inverse transform returns the original data. For convenience, the FFT class has a member function \texttt{setDirectionName} which allows you to associate a character string with each of the transform directions + 1 and -1. You might choose to refer to these directions as "xtok" and "ktox", for example.

In the case of a complex-to-complex FFT or a pure sine transform; the input and output fields are the same size and of the same type. In these instances, we offer the option of performing the transform "in place"; that is, using just one \texttt{Field} argument for both the input and output. For example, we could perform an inverse complex-to-complex FFT with the code
\begin{smallcode}
ccfft.transform(-1,complexField2); 
\end{smallcode}



\subsection{Improving FFT Performance}
Some improvement in performance of the transform method may be obtained by careful selection of the axis ordering of input and output \texttt{Field}s. In order to perform a parallel FFT along a particular dimension, the FFT object will first reorder the axes so that the first axis is the one to be transformed. It does this by assigning the field data into a new \texttt{Field} with a domain in which the order of the original \texttt{Index} objects has been permuted. This new \texttt{Field},
which is maintained internally by the FFT class, has a data layout that is serial along this first dimension and parallel along all other dimensions. With this layout, each processor can independently perform FFTs along the serial axis for each of the one-dimensional strips of data it owns. To subsequently transform along another dimension, the FFT object must again transpose the data so that the next dimension to be transformed is now the first dimension and is serial. These data
transposes can be fairly costly to perform. We can eliminate at least one data transpose if the output \texttt{Field} supplied by the user has the same layout characteristics needed for the final transform (or, in the case of an "in place" transform, if the input \texttt{Field} matches the layout needed for either the first or last transform), and has no guard cell layers. For instance, let us assume we have a three-dimensional \texttt{Field} of complex numbers and we want to transform
all dimensions. If the \texttt{Index} objects {\tt I,J}, and {\tt K} describe the first, second, and third axes of our \texttt{Field} domain, we could perform a forward FFT with the line
\begin{smallcode}
ccfft.transform(+l,complexFieldl);
\end{smallcode}

If the first dimension of \texttt{complexField} is serial, the transform method will skip the first data transpose because the input data is already distributed appropriately for transforming along the first dimension. Similarly, if we were to call an inverse transform with this same \texttt{Field}, it would transform the axes in reverse order, and we would be able to skip the final data transpose. Alternatively, we might choose to do this FFT using separate input and output \texttt{Field}s:
\begin{smallcode}
ccfft.transform(+1,complexField1,complexField2); 
\end{smallcode}
In this case, the final optional argument to the "two-field" trans form function defaults to false, meaning that \texttt{complexFieldl} is not considered constant and may be used in place of a temporary \texttt{Field} to avoid the first data transpose. In addition, the output \texttt{Field} can be used in place of the final temporary \texttt{Field} if it has the proper layout. If \texttt{complexField2} has its axes reordered so that its first axis is the final axis to be transformed
(e.g., \texttt{K}, then \texttt{I}, then \texttt{J}) and that first axis is serial, then we can skip the final data transpose. This choice of data layout results in a slightly faster parallel FFT, and it is often convenient if all you need to do is transform the data, do a brief computation with the transformed data, and then invert the transform.

Another issue of relevance to the performance of the transform method is the type of routine used to perform the actual one-dimensional FFT. Currently, we provide two options for this. The first is Fortran 77 implementations of FFT routines from the Netlib repository. These are portable and highly optimized routines that we invoke via C++ wrapper functions. The second option (available only on SGI and Cray systems) is native FFT routines from the SGI/Cray Scientific Library. These routines can be substituted for the portable Netlib routines by supplying the option {\tt USE\_SCSL\_FFT} to the configure utility before compiling the \ippl library. These native library routines tend to be somewhat faster than the portable Fortran routines, and we plan to offer the ability to use native FFT routines such as FFTW in the future. 
