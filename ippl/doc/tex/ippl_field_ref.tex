\chapter{Fields}
\label{sec:fields}
The Field class represents the common computational science abstraction of a continuum (mathematical) field discretized on a mesh, 
with some centering on that mesh. Used in it simplest, default, way with basic-type elements such as double's or float's. Field serves as a 
multidimensional array class. Similarly, you can use collections of simple or not-so-simple instances of the Field class to represent almost any 
data structure found in computational science; but this is against the philosophy of object-oriented design, in which you should design classes 
to represent the physics/numerical-mathematics abstractions of the problem domain, rather than the data structures typically found in computer codes. 
\section{Field Class Definition}
The Field class is parameterized on the type  \texttt{T} of elements (typically a type like  \texttt{double,Vektor}, or \texttt{Tenzor}), dimensionality, mesh type, and centering on the mesh:
\begin{smallcode}
template<class T, unsigned Dim, 
                  class Mesh=Cartesian, 
                  class Centering=Mesh::DefaultCentering> 
class Field : public FieldBase 
\end{smallcode}
\pagebreak
\section{Field Constructors}
To instantiate a \texttt{Field} we use the following of  Field constructors:
\begin{smallcode}
Field(FieldLayout<Dim>&); 
Field (FieldLayout<Dim>&, const GuardCellSizes<Dim>&); 
Field (FieldLayout<Dim>&, const BConds<T,Dim,Mesh,Centering>&); 
Field (FieldLayout<Dim>&, const GuardCellSizes<Dim>&, 
			const BConds<T,Dim,Mesh,Centering>&); 
Field (FieldLayout<Dim>&, const BConds<T,Dim,Mesh,Centering>&, 
			const GuardCellSizes<Dim>&); 
\end{smallcode}
\section{Field Member Functions and Member Data}

\begin{smallcode}
IndexingField<T,Dim,l,Mesh,Centering> operator[] (const Index& idx) 
IndexingField<T,Dim,l,Mesh,Centering> operator[] (int i) 
const iterator& begin() const { return Begin; } 
const iterator& end() const { return End; } 
void fillGuardCells() ; 
const GuardCellSizes<Dim>& getGuardCellSizes() { return Allocated;  

// Boundary condition handling. 
unsigned leftGuard(unsigned d) 	{ return Allocated.left(d); } 
unsigned rightGuard(unsigned d) 	{ return Allocated.right(d); } 
const Index&: getIndex(unsigned d) { return Layout~>get_Domain() [d]; 
const NDIndex<Dim>& getDomain() 	{returnLayout->get_Domain();} 

// Definitions for accessing boundary conditions. 
typedef BCondBase<T,Dim,Mesh,Centering> bcond_value; 
typedef BConds<T,Dim,Mesh,Centering> bcond_container; 
typedef bcond_container::iterator bcond_iterator; 
bcond_value& getBCond(int bC);
bcond_container& getBConds(){return *BC;} 
\end{smallcode}

\section{Operations on Field Objects}
\subsection{Assignment}
For the special case where there is only one term on the right-hand side of an assignment, the assignment operator can be utilized. 
Examples of single term assignments include: 
\begin{smallcode}
unsigned Dim = 2,int N = 100; 
Index I (N), J (N) ; 
FieldLayout<Dim> layout(I,J)i Field<double,Dim> A(layout) , B(layout);
A = 2.0;
B = A; 
\end{smallcode}
For cases where more than one term exists on the right hand side of an assignment, the \texttt{assign()} call must be made. Any combination of scalars, \texttt{Field}'s,  \texttt{IndexingField}'s, and\texttt{ Index}'s
can be put as the second argument of the \texttt{assign()} call. The only requirement in combining terms is that the appearance of an\texttt{ Index} object anywhere inside of an expression requires, 
that all the \texttt{Field} contained in the expression must be indexed. It is not possible to combine \texttt{Field}'s and \texttt{IndexingField}'s in a single expression. 
Nor is it possible to combine \texttt{Field}'s and \texttt{Index} objects in a single expression. The following examples define legal expressions: 
\begin{smallcode}
unsigned Dim = 2;
int N = 100; 
Index I(N), J(N); 
FieldLayout<Dim> layout(I,J); 
Field<double,Dim> A(layout) , B(layout) , C.(layout);

assign(A, 2.0 + B); 
assign(B , A + 2.0); 
assign(B[I][J],3.0+B[I][J]); 
assign(A[I][J] , I + A[I][J]/C[I][J]); 
\end{smallcode}

The following are examples of illegal expression:
\begin{smallcode}
B[I][J] = 3.0 + B[I][J]; 	// must use assign() with indexed Field~s, 
assign(A, 2.0 + B[I][J]); // can't combine indexed Band non~indexed B 
assign(B[I][J] , A + 2.0); 	// can't combine indexed B and non-indexed A 
assign(A[I][J] , I + A/C[I][J]); // can't combine indexed C ,and non-indexed A 
\end{smallcode}

\subsection{Boundary Conditions}
\ippl pre-defines classes to represent 7 different forms of boundary conditions: 
\begin{enumerate}
\item Periodic boundary condition: \texttt{PeriodicFace}
\item Positive reflecting boundary condition: \texttt{PosReflectFace} 
\item Negative reflecting boundary condition: \texttt{NegReflectFace} 
\item Constant boundary condItion: \texttt{ConstantFace} 
\item Zero boundary condition (special case of constant): \texttt{ZeroFace} 
\item Linear extrapalation baundary condition: \texttt{ExtrapolateFace}
\item none (should not be used) 
\end{enumerate}
Let's examine each boundary condition as applied to. the same shift operation. In each case, the first \texttt{assign()} invocation shows how the \texttt{Field A} with the following values: 
\begin{center}
        \begin{tabular}{|c|c|c|c|}
        \hline
        0 & 1 & 2 & 3 \\        \hline
        1 & 2 & 3 & 4 \\        \hline
        2 & 3 & 4 & 5 \\        \hline
        3 & 4 & 5 & 6 \\        \hline	
        \end{tabular}
   \label{tbl:t1}
   \end{center}
The results after the second \texttt{assign()} invocation show how the boundary conditions on \texttt{A} affect the calculation. 
For the first example, consider the case where each boundary of the \texttt{Field} object \texttt{A} has periodic boundary conditions: 

\begin{smallcode}
unsigned Dim = 2; 
Index 1(4), J(4); 
BConds<double,Dim> bc;
 
bc[0] = new PeriodicFace<double,Dim>(0); 
bc[1] = new PeriodicFace<double,Dim>(1); 
bc[2] = new PeriodicFace<double,Dim>(2);
be[3] = new PeriodicFace<double,Dim>(3);

FieldLayout<Dim> layout(I,J); 
Field<double, Dim> A( layout, GuardCellSizes<Dim> (1) , bc) ; 
Field<double,Dim> B(layout); 
assign(A[I][J] ,I + J); 
assign(B[I][J] ,A[I+l][J+l]) 
\end{smallcode}
This code segment produces the following values in the \texttt{Field B}: 
\begin{center}
        \begin{tabular}{|c|c|c|c|}
        \hline
        2 & 3 & 4 & 1 \\        \hline
        3 & 4 & 5 & 2 \\        \hline
        4 & 5 & 6 & 3 \\        \hline
        1 & 6 & 3 & 0 \\        \hline	
        \end{tabular}
   \label{tbl:t2}
   \end{center}
In the case af the periodic boundary conditions, we see that the values wrap around the domain of the \texttt{Field} and pull values from the opposite side of the \texttt{Field} when the 
indexing operations reference positions outside the domain. Note that the specification of a boundary condition above overrides the default behavior, which places zeroes into
positions that attempt to obtain data from outside the domain. 

Next we consider the case where positive reflecting boundary conditions are applied to each 
boundary of the  \texttt{Field} object  \texttt{A} in the same shift operation: 
\begin{smallcode}
bc[0] = new PosReflectFace <double,Dim>(0); 
bc[1] = new PosReflectFace <double,Dim>(1); 
bc[2] = new PosReflectFace <double,Dim>(2);
be[3] = new PosReflectFace <double,Dim>(3);

assign(A[I][J] ,I+ J); 
assign(B[I][J] ,A[I+1][J+1);
\end{smallcode}
This code segment produces the following values in \texttt{Field B}: 
\begin{center}
        \begin{tabular}{|c|c|c|c|}
        \hline
        2 & 3 & 4 & 4 \\        \hline
        3 & 4 & 5 & 5 \\        \hline
        4 & 5 & 6 & 6 \\        \hline
         4 & 5 & 6 & 6 \\        \hline	
        \end{tabular}
   \label{tbl:t3}
\end{center}
In the case of the positive reflecting boundary conditions, we see that the values are simply reflected across the boundary over which the \texttt{Field} indexing operation occur. 
This boundary condition is meant to represent Neuman boundary conditions in physical systems.
 
Next, consider the case where each boundary of the \texttt{Field} object \texttt{A} has negative reflecting boundary conditions:
\begin{smallcode}
bc[0] = new NegReflectFace <double,Dim>(0); 
bc[1] = new NegReflectFace <double,Dim>(1); 
bc[2] = new NegReflectFace <double,Dim>(2);
be[3] = new NegReflectFace <double,Dim>(3);

assign(A[I][J] ,I+ J); 
assign(B[I][J] ,A[I+1][J+1]);
\end{smallcode}
This code segment produces the following values in \texttt{Field B}: 
\begin{center}
        \begin{tabular}{|c|c|c|c|}
        \hline
        2 & 3 & 4 & -4 \\        \hline
        3 & 4 & 5 & -5 \\        \hline
        4 & 5 & 6 & -6 \\        \hline
         -4 & -5 & -6 & -6 \\        \hline	
        \end{tabular}
   \label{tbl:t4}
\end{center}
In the case of the negative reflecting boundary conditions, we see that the values are simply reflected across the boundary over which the \texttt{Field} indexing operation occur and negated. 
This boundary condition is meant to represent Dirichlet boundary conditions in physical systems. 


Next, consider the case where each boundary of the \texttt{Field} object \texttt{A} has constant boundary conditions: 
\begin{smallcode}
bc[0] = new ConstantFace <double,Dim>(0,9.0); 
bc[1] = new ConstantFace <double,Dim>(1,9.0); 
bc[2] = new ConstantFace <double,Dim>(2,9.0);
be[3] = new ConstantFace <double,Dim>(3,9.0);

assign(A[I][J] ,I+ J); 
assign(B[I][J] ,A[I+1][J+1]);
\end{smallcode}
Note additional argument to the \texttt{constantFace} constructor. This argument represents the value which is to be fixed on the boundary in that direction. 
The code segment above produces the following values in \texttt{Field B}: 
\begin{center}
        \begin{tabular}{|c|c|c|c|}
        \hline
        2 & 3 & 4 & 9 \\        \hline
        3 & 4 & 5 & 9 \\        \hline
        4 & 5 & 6 & 9 \\        \hline
         9 & 9 & 9 & 9 \\        \hline	
        \end{tabular}
   \label{tbl:t5}
\end{center}
In the case of the constant boundary conditions, we see that a fixed value is shifted into the domain across the boundary over which the \texttt{Field} indexing operation occur. 

Finally, consider the case where the \texttt{Field} object \texttt{A} has mixed boundary conditions, boundary-by-boundary (i.e., face-by-face): 
\begin{smallcode}
be[0] = new PeriodicFace <double,Dim>(0); 
be[1] = new PeriodicFace <double,Dim>(1); 
be[2] = new PosReflectFace <double,Dim>(2); 
be[3] = new NegReflectFace <double,Dim>(3);
assign(A[I][J] ,I+ J); 
assign(B[I][J] ,A[I+1][J+1]);
\end{smallcode}
The code segment above produces the following values in \texttt{Field B}: 
\begin{center}
        \begin{tabular}{|c|c|c|c|}
        \hline
        2 & 3 & 4 & 1 \\        \hline
        3 & 4 & 5 & 2 \\        \hline
        4 & 5 & 6 & 3 \\        \hline
         -4 & -5 & -6 & -3 \\        \hline	
        \end{tabular}
   \label{tbl:t6}
\end{center}


