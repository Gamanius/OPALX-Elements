\chapter{Index Class}
\label{sec:index}
The \texttt{Index} class represents a strided range of integer indices (described by base, bound, and stride integer values). 
You use it to define the size (index extent) per dimension of \texttt{Field} objects on construction and to specify subsets of 
\texttt{Field} elements along a dimension in \texttt{Field} expressions.
 
It is important to note that the actual memory address of an \texttt{Index} object is relevant to whether that object may be used 
interchangeably with another \texttt{Index} object specifying the same index values. In fact, two \texttt{Index} objects specifying 
the same index values are not interchangeable. This is qualitatively different than the semantics of Fortran 90 array-syntax, 
for example. You may construct a "new" \texttt{Index} object with a different name, but which has the same values and 
is interchangeable with a given \texttt{Index} by setting it equal to the first Index object on construction. 
\section{Index Definition}
\label{sec:indpubldef}
\begin{smallcode}
class Index { 
public: 
// Public data member -- iterator class: class iterator 
{ 
public:
iterator() : Current (0) , Stride(0) {} 
iterator(int current, int stride=l) : Current (current) , Stride(stride){} 
int operator*(); 
iterator operator--(int); // Post decrement iterator& operator--(); 
iterator operator++(int); // Post increment iterator& operator++(); 
iterator& operator+=(int); 
iterator& operator-=(int); 
iterator operator+(int) const; 
iterator operator-(int) const; 
int operator[] (int); 
bool operator==(const iterator &) const; 
bool pperator<const iterator &) const; 
bool operator!=(const iterator &) const; 
bool operator> (const iterator &) const; 
bool operator<=(const iterator &) const; 
bool operator>=(const iterator &) const; 
private: 
int Stride; 
intCurrent; 
};

// Public member functions. 
// Constructors: 
Index ( ) ;                                  // null range
inline Index(unsigned n);      // [0 .. n-1]  
inline Index(int f, int l);         // [f .. l] 
inline Index(int f, int l, int s); // First to Last using Step. 

// Destructor 
~Index () {};         // Don't need to do anything. 

int id () { return 1 ; }
 
inline int min() const;		// the smallest element.
inline int max() const;	// the largest element.
inline int length() const; 	// the number of elems.
inline int stride() const; 	// the stride. 
inline int first() const; 	// the first element. 
inline int last() const; 	// the last element.
inline bool emty() const;	 // is it empty? 
inline const Index* getBase() const; // the base index 
 
// Additive operations. 
friend inline Index operator+ (const Index&, int);
friend inline Index operator+(int,const Index&);
friend inline Index operator~(const Index&,int);
friend inline Index operator-(int,const Index&);

// Multiplicative operations. 
friend inline Index operator-(const'Index&);
friend inline Index operator*(const Index&,int);
friend inline Index operator*(int,const Index&); 
friend inline Index operator I (const Index&,int);

// Intersect with another Index. 
Index intersect(const Index&) const;

// Plug the base range of one into another. 
Index plugBase(const Index &) const;
 
// Test to see if two indexes are from the same base. 
inline bool sameBase(const Index &) const;
 
// Test to see if there is any overlap between two Indexes. 
inline bool touches (const Index& a) const;
 
// Test to see if one contains another. , 
inline bool contains(const Index& a) const;

// Split one. into two. 
inline bool split(Index& I, Index& r) const; 

// iterator begin 
iterator begin() { return iterator(First,Stride); } 

// iterator end 
iterator end () { return iterator (First+Stride*Length, Stride);
 
// An operator< so we can impose some sort of ordering. 
bool operator< (const Index& r) const;

// Test for equality. 
bool operator==(const Index& r) const;
 
static void findPut(const Index&,const Index&, const Index&, Index&, Index&); 

// put data into a message to send to another node 
Message& putMessage(Message& m);

// get data out from a message 
Message& getMessage(Message& m); 

// Print it out. 
friend ostream& operator<<(ostream& out, const Index& I); 
}; 
\end{smallcode}


\section{Index Constructors}
The constructor for \texttt{Index} takes one, two, or three in arguments. In the case of three arguments, 
these represent the base index value, the bounding index value, and the stride. The two and one-argument cases are simplifications. 
\begin{smallcode}
Index I(8);
\end{smallcode}
instantiates an \texttt{Index} object representing the range of integers from 0 through 7 (i.e" [0,7]) with implied stride 1. The two-argument
\begin{smallcode}
Index J(2,8); 
\end{smallcode}
instantiates an \texttt{Index} object representing the range of integers [2,8] with implied stride 1. The three-argument 
\begin{smallcode}
Index(1,8,2); 
\end{smallcode}
instantiates an Index object representing the range of integers [1,8] with stride 2- .. that is, the  ordered set ${1,3,5,7}$.

Note that the single argument in the one-argument case defines the number of elements, rather than the bound. This means that \texttt{Index J(8)} , which represents $[0,7]$, 
is different than \texttt{Index J(0,8)} \texttt{and Index J(0,8,1)} , which both mean $[0,8]$.
 
There is also a special special constructor taking no arguments; this is meant for use in constructing arrays of\texttt{ Index}'s.

\section{Index Member Functions and Member Data}
\subsection{Index iterator} 
The only public member data is the \texttt{Index::iterator} class. This class has the semantics of an STL random access iterator; the advanced user can use 
it to iterate over integer index values represented by the containing \texttt{Index} object. 
The STL semantics means that the class provides increment and decrement operators, increment/decrement by specified integer amounts, deference operator, 
random-access \texttt{operator[]}, and comparison operators. See the class definition in Section \ref{sec:indpubldef} for the full list. The \texttt{iterator class} definition is contained within. 

\begin{smallcode}
iterator begin () 
\end{smallcode}
Returns an \texttt{Index: : iterator} positioned at the beginning of the index values represented by the \texttt{Index} object.
\begin{smallcode}
iterator end ( ) 
\end{smallcode}
Returns an \texttt{Index::iterator} positioned beyond the last index value represented by the \texttt{Index} object. 


\subsection{Index Query/Accessor Functions} 
These functions mostly return values of or values computed from private data members of the \texttt{Index}. 
\begin{smallcode}
int min () const 
\end{smallcode}
Returns the smallest integer value allowed for the \texttt{Index} object. 
\begin{smallcode}
int max () const 
\end{smallcode}
Returns the largest integer value allowed for the \texttt{Index} object. 
\begin{smallcode}
int length() const 
\end{smallcode}
Returns the total number of integer values spanned by the \texttt{Index} object. 
\begin{smallcode}
int stride()const
\end{smallcode} 
Returns the value of the stride for the \texttt{Index} object. 
\begin{smallcode}
int first() const 
\end{smallcode}
Returns the integer value ofthe first element in the \texttt{Index} object. 
\begin{smallcode}
int last () const 
\end{smallcode}
Returns the integer value of the last element the \texttt{Index} object. 
\begin{smallcode}
bool empty() const 
\end{smallcode}
Returns true/false depending on whether the \texttt{Index} object is empty. Empty (true) means, that it was constructed with the zero-argument constructor.
\begin{smallcode}
const Index* getBase() const 
\end{smallcode}
Returns a pointer to the base \texttt{Index} object associated with the \texttt{Index} object. As described at the beginning of this chapter, the memory address of an \texttt{Index} 
is important, and if two \texttt{Index}'s are to be interchangeable they must share the same base address as well as be conforming (have the same base, bound, and stride values). 


\subsection{Index Arithmetic operator Functions} 
Here we describe the arithmetic operator functions defined in the Index class to act on \texttt{Index} objects. In the example code in these descriptions, \texttt{I} is an \texttt{Index} object and \texttt{n} is an \texttt{int}.
\begin{smallcode}
friend Index operator+(const Index&,int)
\end{smallcode} 
The \texttt{Index} expression\texttt{I +n} invokes this operator. It adds an integer value \texttt{n} to the base and bound values of \texttt{I}, then constructs and returns the resulting \texttt{Index} object using those 
	revised values.
 
\begin{smallcode}
friend Index operator+(int,const Index&) 
\end{smallcode}
Same as the previous, except for the order of the operands. That is, the \texttt{Index} expression \texttt{n+I} invokes this operator. 

\begin{smallcode}
friend Index operator-(const Index&,int) 
\end{smallcode}
Invoked by \texttt{ I-n} subtracts \texttt{n} from the base and bound of \texttt{I} and returns the resulting \texttt{Index}. 
\begin{smallcode}
friend Index operator-(int,const Index&) 
\end{smallcode}
Invoked by \texttt{n-I} subtracts the base of \texttt{I} from \texttt{n}, multiplies the stride by \texttt{ -1},and returns the 
	resulting \texttt{Index} object.

\begin{smallcode}
friend Index operator-(const Index&) 
\end{smallcode}
Negation operator, invoked by \texttt{-1}. This multiplies the base and bound of \texttt{I} by \texttt{-1} and returns the resulting Index object. 
\begin{smallcode}
friend Index operator*(const Index&,int) 
friend Index operator*(int,const Index&) 
\end{smallcode}

Invoked by \texttt{I*n} and \texttt{n*I}, respectively. These multiply the base, bound, and stride of \texttt{I} by \texttt{n} and return the resulting \texttt{Index} object 
\begin{smallcode}
friend Index operator /(const Index&,int) 
\end{smallcode}
Invoked by \texttt{I/n}. These divide the base, bound, and stride of \texttt{I} by \texttt{n} and return the resulting \texttt{Index} object (integer division, truncates). 
Note that the division operator with the operands the other way around (\texttt{n/I}) is not defined.

\subsection{Index I/O and Message-Passing Functions}
These are functions are to write out an \texttt{Index} object, and to pack/unpack and send/receive an \texttt{Index} object as a message between two processes. 
\begin{smallcode}
friend ostream& operator<<(ostream&, const Index&) 
\end{smallcode}
Formatted insertion of the contents of an \texttt{Index} object into the output stream. The interface is the stream I/O  \texttt{operator <<}, invoked by \texttt{os<<I} (os is an ostream object, or a \ippl Inform object).

\begin{smallcode}
static void findPut(const Index&, const Index&, const Index&,Index&, Index&)
\end{smallcode} 
Need to describe .....
  
\begin{smallcode}
Message& putMessage(Message&)
\end{smallcode} 
Put \texttt{Index} data into a message to send to another node. 
\begin{smallcode}
Message&getMessage{Message&) 
\end{smallcode}
Get  \texttt{Index} data out of a message you have received from another node. 

\subsection{Index Comparison Operators} 
\begin{smallcode}
bool operator < const Index&) const;
\end{smallcode} 
A less-than test so we can impose some sort of ordering of two \texttt{Index} objects. The implementation is such that \texttt{I<J} returns true if one of the following is true: 
the length of \texttt{I }(number of integer index values represented by \texttt{I}) is less than the length of \texttt{J}, the length's are, equal but the first integer index value in 
\texttt{I} is less than the first value in \texttt{J}, or (failing either of the first two tests, and given that the length of \texttt{I} is greater than 0) the stride of I is less than the stride of \texttt{J}. 
\begin{smallcode}
bool operator == (const Index&) const 
\end{smallcode}
Test for equality of two \texttt{Index} objects.\texttt{I==J} returns true if the {base,bound,stride} values of \texttt{I} and \texttt{J} are the same. This does not check that the two 
\texttt{Index} objects have the same base address . 


\subsection{Index Composition Functions}
\begin{smallcode}
Index intersect(const Index&) const
\end{smallcode} 
\texttt{I.intersect(J)} returns an \texttt{Index} object containing the intersection of \texttt{I} with \texttt{J}  such that it contains all the integer index values contained both in \texttt{I} and \texttt{J} (expressed as the 
	{base,bound,stride} of the new \texttt{Index} object).
\begin{smallcode}
Index plugBase(constIndex&) const
\end{smallcode} 
Plug the base range of one into another. 
\begin{smallcode}
inline bool sameBase(const Index&) const 
\end{smallcode}
Test to see if two \texttt{Index}'s are from the same base. Internally, \texttt{Index} contains a (private) \texttt{Index*} which points to the base \texttt{Index} object from which it was constructed. If it is an \texttt{Index}
which was constructed explicitly with one of the constructors described in Section xxx then this will be a pointer to itself. If you construct an \texttt{Index} object by arithmetic on an existing, object, the pointer points to, the existing Index. Example:
\begin{smallcode}
Index I(10);
Index J( 10); 
bool t1 = I.sameBase(J);     // false 
bool t2 = I.sameBase(I+1); // true 
\end{smallcode}
\begin{smallcode}
inline bool touches(const Index&) const 
\end{smallcode}
Test to see if there is any overlap between two Indexes. \texttt{I.touches(J)} returns true if the minimum integer value represented by \texttt{I} is less than or equal to the maximum value represented by
\texttt{J}, and the maximum integer value represented by \texttt{I} is greater than or equal to the minimum value represent by \texttt{J}. 
\begin{smallcode}
inline bool contains(const Index&) const
\end{smallcode}
Test to see if one \texttt{Index} object completely conntains another. \texttt{I.contains (J)} returns true if the minimum integer value represented by \texttt{I} is less than or equal to the minimum value represented by \texttt{J}, and the maximum integer value represented by \texttt{I} is greater than or equal to the maximum value represented by \texttt{J}. 
\begin{smallcode}
inlinebool split(Index& 1, Index& r) const 
\end{smallcode}
Splits one \texttt{Index} object into two. \texttt{I.split(J,K) }divides the set of integers represented by \texttt{I} into two halves, and fills the \texttt{Index} object \texttt{J} with the left half of the set, and the \texttt{Index} object \texttt{K} 
with the right half of the set. 












